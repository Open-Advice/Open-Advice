\chapter{We're Not Crazy \dots We're Conference Organizers! - Gareth J.
Greenaway}

\textit{Gareth J. Greenaway has been actively involved in the Free \& Open Source community since 1997 after discovering Linux. A majority of this involvement has been gathering like-minded people to learn and experience new elements of Free \& Open Source software. This involvement began with a small Linux Users Group and has expanded into organizing the Southern California Linux Expo, also known as SCALE. As one of the founding members of the event, Gareth current holds two key positions with the organization. The first role is Conference Operations and the second is Community Relations.}

I started writing this section with what I saw as the requirements and steps for
organizing a Free \& Open Source conference, however, most of what I found myself saying had been covered by community management expert Dave Neary. So rather than repeat and overlap what Dave had to say, I decided to share various stories from organizing SCALE along with lessons that were learned over the years.

\section*{Too much power!}

SCALE was started 9 years ago by members of three local Linux Users Group,
growing out of a small regional event organized by one of these LUGs. The first
time around was definitely a learning experience. Many lessons were learned,
there was quite a bit of running around and the event seemed to fly by very
quickly. Because none of us had planned an event where we had to be
concerned about the load on electrical circuits or power usage, we had not
considered it and because of that we ended up tripping the electrical breakers
for the venue several times throughout the event.

\section*{It’ll work \dots its wireless!}

The second SCALE included many of the lessons learned from the previous year but
a new venue would result in new lessons. The Los Angeles Convention Center
served as the location for SCALE 2, providing much more space to spread out for
the event. The new location also served as our first lesson in contracts with a
large organization for things such as A/V equipment, Internet access and
exhibitor furniture.  

Because of the placement for the event within the convention center, we were
forced to locate the shows registration counters in an area that while visible
to arriving attendees would be some distance away from the rest of the show. 
Our options for providing network access to the registration area was limited as
fire regulations prevented running wire, so wireless was the only option. Everything was setup early the day for the show and was working great until mysteriously it was not. The wireless connection providing the much needed network access to the registration counter would simply disappear. There was much troubleshooting, much relocation of equipment and antennas and much frustration. ``It should be working'' was the only conclusion that everyone could come to, with little insight into why it simply was not working. Suddenly one of the team members, who had been standing some distance from the troubleshooting session, called everyone over to where he had been standing. In front of a large window which overlooked a large convention hall on the lower level, suddenly we all saw what it was he wanted us to see. Below us where dozens of flashing, spinning, pulsating lights staring up at us and almost mockingly.\todo{What was the light?} 
We suddenly realized that our hours of working, attempting to solve this
wireless issue had been futile. In the end we ran an Ethernet cable, taped it down securely as best we could and said a small prayer that the fire marshal
would not make a surprise inspection.

\section*{Awards shows, snipers and the case of the missing IBM case}

Perhaps one of the well-known stories from the history of SCALE is the mishaps
and adventures that took place at SCALE 3. The adventures are well-known because
as a SCALE attendee that year you could not help but experience them.

The third SCALE was set to take place once again at the L.A. Convention Center,
the many months of planning and prep work had been done and everything was
shaping up nicely. About 3 weeks prior to the event we received some
information about various road closures around the convention center because of
an upcoming awards show. The road closures resulted in there being one way in
and out of the convention center, definitely not the ideal situation. 
Fortunately we had the time to alert everyone coming out for the show about the
road closures and alternative routes. This was also the first year that SCALE
would be a 2-day show, the hope being that things would be spread out a bit and
not feel as rushed and hectic.

One of the long standing sponsors and exhibitors that SCALE has had over the
years is IBM. They have always remained a welcome addition to the show,
unfortunately their attendance is also usually met with some difficulty. The
day before the event has typically been reserved as a setup day, an opportunity
for the SCALE staff to setup and exhibitor to prepare their booths. It is also
the day that any packages that exhibitors have delivered arrive. IBM had
planned to showcase a new server line at the show and had had one of these
servers shipped to the convention center, unfortunately it had not been
delivered to their booth and no one at the convention center knew the
whereabouts of the package. Many hours of searching all the possible locations
within the convention center had not turned up any clues.

As it turned out, the awards show that would be taking place in a few days had
rented a number of rooms for office space and storage needs. On a whim, the
event coordinator who was assisting in the search suggested perhaps we search
one of their storage rooms in hopes that the IBM case had been delivered there
accidentally. The room in question was a small storage closet, inside we found
boxes and boxes from the floor to the ceiling of tickets for the upcoming awards
show. Behind these boxes, off in a corner was a large blue case with the IBM
logo printed across it. Crisis averted!

The rest of the event ran smoothly and was relatively incident-free. As the
event wound down a small crowd began to form near some large windows overlooking
the street outside, as I walked past I realized what it was everyone was looking
at. Several figures, all dressed in black, were moving around on the rooftops of
the buildings across the street. All of these figures were carrying sniper
rifles and were members of the Los Angeles Police Department’s SWAT team, there
in preparation for the awards show that would be starting a few hours from then.
This definitely made for an exciting departure from the convention center.

\section*{No room at the inn}

The fourth SCALE resulted in another venue change, this time the switch was to a
hotel instead of a convention center. As the years went by, more and more people
were traveling to attend SCALE and staying at local hotels, we decided to
explore the possibility of holding SCALE in a hotel. We scouted the area and
ended up working with an event coordinator on find the right venue for the
event. Settling on a hotel near the Los Angeles airport, the planning began. 
Holding an event at the hotel quickly became a source for new lessons on dealing
with factors unique to a hotel. One of the most important lessons that we came
to learn was making sure that all contracts had an agreed-upon cancellation
policy.

Roughly five weeks prior to the event we received a call from the venue telling
us that their corporate entity was canceling our event and giving our space to
another event. Obviously this came as quite a shock and left us scrambling. 
The contract with the hotel did not include any sort of agreement for
relocation, but simply stated that they could cancel the event without cause.

After many phone calls and negotiations with the original venue, eventually they
were willing to provide some funds to help relocate to another venue. The new
venue was also willing to honor the same terms regarding electrical, Internet
access and A/V equipment. Everything worked out and the SCALE team had learned
a valuable lesson when negotiating future contracts.

\section*{Curtain Call}

All in all, organizing a conference is a rewarding endeavor and a great way to
give back to the community. Conferences are an important element, they allow in
person interaction in a world that commonly relies on virtual means of
communication.

Advice I would give to future conference organizers would be:
\begin{itemize}
 \item Start small, do not cram too much into an event the first year.
 \item Take chances, make mistakes, do not be afraid to fail.
 \item Communication is key!
\end{itemize}
