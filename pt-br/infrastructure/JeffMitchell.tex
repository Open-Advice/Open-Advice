\chapterwithauthor{Jeff Mitchell}{Ame o desconhecido}

\authorbio{Jeff Mitchell passa seus dias de trabalho envolvido com todos
os tipos de tecnologias da computação, e seus dias de folga com todos os
tipos de projetos de Software Livre e a maioria se beneficia de uma confluência
de ambos. Depois de atuar profissionalmente como um administrador de sistemas
entre 1999-2005, ele tem mantido suas habilidades afiadas trabalhando
voluntariamente para vários projetos de Software Livre e outros tipos de trabalho.
Hoje, a maioria do seu tempo em Software Livre é gasto como administrador de
sistemas para o KDE e desenvolvedor principal do \textit{player} Tomahawk.
Jeff atualmente vive em Boston, EUA.}

\noindent{}Recentemente eu fiz parte de um grupo para entrevistar um potencial
novo administrador de sistemas no trabalho. Nós tínhamos passado por algumas
dezenas de currículos e finalmente trazíamos o nosso primeiro candidato para
uma entrevista. O candidato -- vamos chamá-lo de John -- tinha experiência
com \textit{clusters} de computadores menores, estilo laboratório, assim como
com operações maiores de centro de dados. No início, as coisas estavam correndo
bem, exceto por suas respostas estranhas para algumas de nossas perguntas: ``Eu
sou um administrador de sistemas.'' O significado dessa afirmação não ficou
imediatamente claro para nós, até o seguinte diálogo acontecer:
\begin{quote}
\textbf{Eu}: Então, você disse que não tem experiência com Cisco IOS,
e quanto a redes em geral?\newline
\textbf{John}: Bem, eu sou um administrador de sistemas.\newline
\textbf{Eu}: Certo, mas -- e sobre os conceitos de rede? Os protocolos
de roteamento como BGP ou OSPF, VLANs, pontes \dots \newline
\textbf{John, irritado}: Eu sou um \emph{administrador de sistemas}.
\end{quote}
Foi quando entendi o que ele estava dizendo. John não estava nos dizendo
que ele sabia de várias coisas que perguntamos porque ele era um amdinistrador
de sistemas; ele estava nos dizendo que por ser um administrador de sistemas
ele \emph{não} sabe sobre essas coisas. John era um administrador de \emph{sistemas};
afirmar isso era sua forma de acenar que essas tarefas pertenciam aos administradores
de redes. Provavelmente sem surpresa, John não conseguiu o trabalho. 

\paragraph*{}Para muitos projetos de Software Livre, especialização é uma maldição,
não uma benção. Se um projeto falha em uma ou outra categoria, geralmente depende
do tamanho da equipe de desenvolvimento; especialização ao grau de pontos únicos de
falhas pode significar sérias perturbações a um projeto no caso de um desenvolvedor
sair, seja em termos bons, ruins ou infelizes. Isso não é diferente para administradores
de sistemas de projetos de Software Livre, embora a escassez geral destes pareça permitir
aos projetos, algumas vezes, adotar tolerâncias perigosas.

O exemplo mais notório que já vi envolveu um projeto em particular cujo site de
documentação (incluindo toda sua documentação de instalação e configuração) esteve
fora do ar por mais de um mês. O motivo: o servidor havia caído, e a única pessoa
com acesso a esse servidor estava navegando em um ``navio pirata'' com membros do
Partido Pirata da Suécia. Isso realmente aconteceu.

Entretanto, nem todos os pontos únicos de falha se devem a administradores de
sistemas ausentes; alguns são artificiais. As decisões de permissão de acesso
de administração do sistema de um grande projeto foram controladas por um único
administrador líder, que não só reservou algumas permissões de acesso apenas para
si (você adivinhou: sim, ele desapareceu por um tempo e sim, isso causou problemas)
mas tomou decisões sobre quanta permissão de acesso seria dada baseada em sua
confiança no candidato. ``Confiança'' nessa caso era baseada em uma coisa; não
era baseada em quantos membros da comunidade apoiaram aquela pessoa, quanto tempo
ela tinha sido um colaborador ativo e confiável para esse projeto, ou mesmo quanto
tempo ele tinha conhecido essa pessoa como membro do projeto. Pelo contrário, foi
baseada em quão bem ele conhecia pessoalmente alguém, o que queria dizer quão bem
ele conhecia aquele indivíduo \emph{em pessoa}. Imagine o quão bem isso funciona
para uma equipe global distribuída de administradores de sistemas.

É claro, esse exemplo só vai mostrar que é muito difícil para os \textit{sysadmins}
de Software Livre andar na linha entre segurança e capacidade. Grandes empresas podem
pagar mais pessoas, mesmo quando esses funcionários são segmentados em diferentes
responsabilidades ou domínios de segurança. Ter mais pessoas é importante,
mas e se a única opção atual para administração de sistema com mais pessoas é pegar
o primeiro cara que aparece aleatoriamente em seu canal IRC e voluntários para
ajudar? Como você pode confiar razoavelmente nessa pessoas, em suas habilidades,
ou em suas motivações? Infelizmente, só os colaboradores de projetos, ou algum
subconjunto deles, podem determinar quando a pessoa certa chegou, usando o mesmo
modelo da Rede de Confiança que fundamenta o mundo do Software Livre. O universo 
dos projetos de Software Livre, suas necessidades e aqueles dispostos a colaborar
para qualquer projeto, é felizmente diversificado; como um resultado, a dinâmica
humana, confiança, intuição e como aplicar esses conceitos a qualquer projeto de
Software Livre, são grandes temas que estão longe do escopo desse pequeno ensaio.

\paragraph*{}Uma coisa principal, no entanto, tem facilitado se manter nessa linha
de segurança/capacidade: o surgimento de sistemas de controle de versão distribuídos,
ou DVCSes. No passado, o controle de acesso era primordial porque o coração de qualquer
projeto de Software Livre -- seu código-fonte -- era centralizado. Sei que muitos por aí
devem estar pensando ``Jeff, você deveria saber melhor que isso; o coração de um projeto
é sua comunidade, não seu código!'' Minha resposta é simples: os membros da comunidade
vêm e vão, mas se alguém acidentalmente executa um ``rm -rf'' em toda a árvore do VCS
centralizada de seu projeto e você não tem \textit{backups}, quantos desses membros da
comunidade estarão dispostos a ficar e ajudar a recriar tudo do zero? (Isso na verdade
é baseado em um exemplo real, onde um membro da comunidade, que estava bêbado, irritou-se
com um código que ele estava depurando e executou um ``rm -rf'' em seu \textit{checkout}
inteiro, \emph{pretendendo} destruir o código todo do projeto. Felizmente, ele não era
um \textit{sysadmin} com acesso ao repositório central, e estava muito bêbado para se
lembrar que sua cópia era simplesmente um \textit{checkout}.)

O código de um projeto é o seu coração; seus membros da comunidade são a sua
força vital. Sem ambos, você terá dificuldades para manter um projeto vivo.
Com um VCS centralizado, se você não tem a prevenção de definir \textit{backups}
regulares, talvez você possa ter sorte e ser capaz de reunir a árvore inteira
de fontes dos \textit{checkouts} que diferentes pessoas tinham de várias partes
da árvore, mas para a maioria dos projetos, a história do código é tão importante
quanto o próprio código atual, e você ainda terá perdido tudo.

Isso não é mais o caso. Quando cada clone local tem toda a história de um projeto
e \textit{backups} nortunos podem ser feitos por ter um comando agendado tão simples
quanto ``git pull'', o repositório centralizado é agora apenas uma ferramenta de
coordenação. Isso diminui um pouco o seu status. Ele ainda tem que ser protegido
contra ameaças internas e externas: sistemas sem correções ainda são vulneráveis
a ataques conhecidos, um \textit{sysadmin} mal-intencionado pode ainda causar estragos,
um sistema de autenticação ineficiente pode ainda permitir código malicioso em sua
base de código, e um acidental ``rm -rf'' de um repositório centralizado pode ainda
causar perda de tempo ao desenvolvedor. Mas esses desafios podem ser superados, e em
tempos de hospedagem barata de VPS e data center, a ausência de \textit{sysadmins}
pode ser superada também. (Melhor garantir que você tenha acesso redundante para DNS!
Ah, e colocar seus sites em um repositório DVCS também, e fazer \textit{branches}
para modificações locais. Você me agradecerá depois.) Assim, os DVCSes dão ao seu
projeto corações extras quase de graça, que é uma grande forma de ajudar os \textit{sysadmins}
de Software Livre a dormirem à noite e fazer-nos sentir um pouco mais como Senhores do Tempo.
Isso significa que se você não está em um DVCS, pare de ler nesse momento e vá mudar para um.
Isso não é apenas sobre fluxo de trabalho e ferramentas. Se você se preocupa com a segurança
de seu código e seu projeto, você mudará.

\paragraph*{}Redundância de código-fonte é uma obrigação, e, em geral,
quanto maior a quantidade de redundância que você puder gerenciar, mais
robustos são os seus sistemas. Pode também parecer óbvio que você queira
redundância de administradores de sistemas; o que você pode não achar óbvio
é que \textit{sysadmins} redundantes não são tão importantes quanto
habilidades redundantes. John, o administrador de sistemas, trabalhou
em data centers e empresas com \textit{sysadmins} redundantes, mas rígidos e
com habilidades definidas. Enquanto isso funciona para grandes empresas,
que podem pagar por novos \textit{sysadmins} com habilidades particulares
sob demanda, a maioria dos projetos de Software Livre não tem esse luxo. 
Você tem que trabalhar com o que tem. Isso, claro, significa que uma
alternativa (e, às vezes, a única alternativa) para encontrar administradores
de sistemas redundantes é espalhando a carga de trabalho, ter outros
membros do projeto assumindo uma competência ou duas até que a redundância
seja atingida.

Isso não é, na verdade, diferente para o desenvolvedor ou para o artista
de um projeto; se metade da sua aplicação é escrita em C++ e outra é em
Python, e somente um desenvolvedor sabe Python, se esse desenvolvedor
deixa o projeto isso causará enormes problemas a curto prazo e sérios
a longo prazo também. Encorajar desenvolvedores a se diversificarem e
se familiarizarem com mais linguagens, paradigmas, bibliotecas, e assim
por diante, significa que cada um de seus desenvolvedores se torna mais
valioso, e isso não deveria vir como um choque; adquirir novas habilidades
é uma consequência da educação contínua, e uma equipe mais educada é mais
valiosa. (Isso também torna os seus CV mais valiosos, o que deve fornecer
um bom impulso.)

A maioria dos desenvolvedores de Software Livre que eu conheço acham um
desafio e um prazer continuar testando novas coisas, já que esse é o
comportamento que os leva ao desenvolvimento de Software Livre em primeiro
lugar. Da mesma forma, administradores de sistemas que trabalham
com Software Livre são escassos e não podem se dar ao luxo de ficarem
presos em uma rotina. Novas tecnologias relevantes para o \textit{sysadmin}
estão sempre surgindo, e há, muitas vezes, formas de usar tecnologias
já existentes, ou tecnologias mais velhas de novas formas, para melhorar
a infraestrutura ou aumentar a eficiência.

John não era um bom candidato porque ele trouxe pouco valor; e ele trouxe
pouco valor porque ele nunca tinha saído de seu papel definido. \textit{Sysadmins}
de Software Livre que caem nessa armadilha não apenas prejudicam o projeto
com o qual eles estão envolvidos, mas eles reduzem seu valor para outros projetos
que usam diferentes tecnologias de infraestrutura e que poderiam desesperadamente
dar uma ajuda; isso diminui a capacidade global da comunidade de Software Livre.
Para o administrador bem-sucedido que trabalha com Software Livre, não existe
essa coisa de zona de conforto.
