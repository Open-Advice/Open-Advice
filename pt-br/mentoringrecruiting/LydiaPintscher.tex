\chapterwithauthor{Lydia Pintscher}{Se Permitindo Fazer Coisas Incríveis}

\authorbio{Lydia Pintscher é interessada em pessoas e criadora de gatos por natureza. Entre outras
coisas, ela gerencia os programas de mentoria do KDE (\textit{Google Summer of Code}, \textit{Google
Code-in} e \textit{Season of KDE}), é membro-fundadora do \textit{KDE Community Working Group}
e é membro do Board de Diretores do KDE e.V.}

\noindent{}O Software Livre tem um inimigo. E não é o que a maioria das pessoas na Internet acha
que é. Não, é a falta de participação ativa.

A cada dia, milhares de pessoas procuram uma forma de dar sentido às suas vidas,
tentando fazer algo que seja verdadeiramente importante. A cada dia, milhares
de linhas código de projetos de Software Livre estão aguardando serem escritas e
depuradas, programas estão aguardando serem divulgados e traduzidos, artes gráficas
estão aguardando serem criadas e muito mais. Infelizmente, na maioria das vezes, pessoas
não conseguem se conectar aos projetos. Existem diversas razões para isso. Tudo começa
com a falta de qualquer conhecimento sobre Software Livre, seus benefícios e propósitos.
Mas nós estamos chegando lá. Pessoas estão começando a utilizar e, talvez até mesmo,
entender o Software Livre. Os projetos de Software Livre vivem da transformação
de alguns destes usuários em colaboradores ativos. E é aí que os problemas começam.

Eu tenho gerenciado centenas de estudantes em programas de mentoria e trabalhado com
promoção, em diversas formas, de projetos de Software Livre. Eu trabalhei com pessoas
entusiasmadas, cujas vidas foram mudadas para melhor através de suas contribuições
ao Software Livre. Mas existe um assunto bastante recorrente que parte meu coração porque
eu sei qual talento estamos perdendo: não se permitir fazer coisas incríveis. O
problema é resumido por uma frase que ouvi de um mentor do Google Summer of Code: "é
difícil constatar que a maioria das pessoas não foi impedida de realizar suas contribuições,
simplesmente não caminharam rápido o suficiente no momento correto". Colaboradores potenciais
frequentemente acham que eles não estão autorizados a contribuir. As razões para isso são
muitas e são todas equivocadas. Os equívocos mais comuns, em minha experiência, são:

\begin{itemize}
 \item "Eu não sei programar. Não há como eu contribuir".
 \item "Eu não sou realmente bom nisso. Minha ajuda não é necessária".
 \item "Eu seria simplesmente um fardo. Eles têm coisas mais importantes a se preocupar".
 \item "Eu não sou necessário. Eles já devem ter pessoas muito mais brilhantes que eu".
\end{itemize}

Tais razões são quase sempre falsas e eu gostaria que eu tivesse sabido, há muito tempo atrás, que
elas eram tão predominantes. Eu teria feito muitos dos meus esforços iniciais de divulgação
de forma diferente.

A forma mais simples de tirar pessoas dessa situação é convidá-las pessoalmente.
"Aquele workshop que estamos fazendo? Claro, você pode participar". "Aquele
\textit{bug}? Eu estou certo que você é a pessoa perfeita para tentar corrigi-lo".
"Aquele comunicado para a imprensa que nós precisamos terminar? Seria ótimo se
você pudesse revisá-lo e ver se está tudo ok". E se isto não for possível, certifique-se
que seu material de divulgação (você tem algum, certo?) afirma claramente que tipo
de pessoa você está procurando e quais são os requisitos básicos. Certifique-se de
atingir pessoas fora da sua base habitual de colaboradores porque para elas esta
barreira é ainda maior. E, ao menos que você supere isso, você irá recrutar somente
aqueles iguais a você -- fazendo com que você tenha mais colaboradores exatamente
iguais aos que você já tem. Pessoas iguais àquelas que você já tem são ótimas,
mas pense em todas as outras excelentes pessoas que você está perdendo e que poderiam
trazer novas ideias e habilidades para o seu projeto.
