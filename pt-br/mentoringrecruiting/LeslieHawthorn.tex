\chapterwithauthor{Leslie Hawthorn}{Um Dia Você Saberá Tudo o que Eles Esqueceram}

\authorbio{\textit{Community manager}, palestrante e autora internacionalmente conhecida, Leslie
Hawthorn tem mais de 10 anos de experiência em gerenciamento de projetos de alta tecnologia, marketing
e relações públicas. Recentemente, ela integrou o AppFog como gerente de comunidade,
onde é responsável pelo envolvimento de novos desenvolvedores. Antes do AppFog, ela
atuou como \textit{outreach manager} no laboratório de Software Livre da Oregon State University
e como \textit{program manager} na equipe de Software Livre do Google, onde gerenciou o
programa Google Summer of Code, criou a competição hoje conhecida como Google Code-in
e lançou o blog Open Source Developer da empresa.}

\begin{quote}
"A documentação mais importante para usuários iniciais são as coisas básicas: como
instalar rapidamente o software, uma visão geral sobre como ele funciona e talvez
algumas instruções sobre como fazer as atividades mais comuns. Estas são exatamente
as coisas que escritores de documentação conhecem bem demais -- tão bem que pode ser
difícil para eles ver as coisas sob o ponto de vista do leitor e cuidadosamente descrever
os passos que (para os escritores) parecem tão óbvios a ponto de serem indignos de menção."
-- Karl Fogel, Produzindo Software Livre.
\end{quote}

Quando você está iniciando seu primeiro trabalho em um projeto de Software Livre,
a curva de aprendizado é íngreme e o caminho intimidador. Você provavelmente estará inscrito
em listas de discussão ou participará de salas de bate-papo com todo o tipo de pessoas "famosas",
como o criador da sua linguagem de programação favorita ou o mantenedor do seu pacote favorito,
imaginando como você se tornará habilidoso o suficiente para contribuir de forma efetiva. O que
você pode não perceber é o quanto essas pessoas sábias esqueceram ao longo do seu caminho para o
sucesso.

Para usar uma analogia simples, o processo de aprender como usar e desenvolver um
projeto de Software Livre é como aprender a andar de bicicleta. Para aqueles que são
ciclistas experientes, "é tão fácil quanto andar de bicicleta". Você provavelmente já
andou de bicicleta algumas vezes e compreende a sua arquitetura: selim, rodas,
freios, pedais e guidão. Mesmo assim, você sobe a bordo, avista o seu passeio e de repente
percebe que andar de bicicleta não é tão simplista quanto você pensou: a qual altura
você deve posicionar o selim? Qual marcha você deve utilizar ao subir uma montanha?
E ao descer? Você realmente precisa daquele capacete? (Dica: sim, você precisa).

Quando você começa a andar de bicicleta, você nem mesmo sabe quais perguntas fazer
e somente saberá após joelhos doendo e algumas pancadas nas costas. Mesmo assim,
nem sempre suas perguntas irão produzir as respostas que você precisa; alguém poderia
te dizer para abaixar o selim ao saber que seus joelhos estão doendo ou simplesmente
assumir que você ainda está aprendendo e irá descobrir as coisas do seu próprio jeito.
Essas pessoas esqueceram como lidar com trocas de marchas, como perceber que estão
utilizando luzes e refletores errados e quais sinalizações indicam uma conversão à
esquerda porque eles têm andado de bicicleta há tanto tempo que tais questões são
simplesmente de natureza secundária para elas.

O mesmo cenário acontece ao iniciar no Software Livre. À medida em que você constrói
um pacote pela primeira vez, você inevitavelmente enfrentará alguma messagem obscura de
erro ou outro tipo de falha. E, ao pedir por ajuda, alguma alma amigável certamente lhe
dirá: "é fácil, simplesmente faça X, Y e Z". Exceto pra você, isto não é realmente fácil:
pode não existir nenhuma documentação sobre X, Y pode não fazer o que se espera que faça
e o que é mesmo essa coisa chamada Z com todas as suas oito entradas de desambiguação
na Wikipédia? Você obviamente não quer ser uma praga mas você irá precisar de ajuda
para realmente conseguir fazer alguma coisa. Talvez você continue refazendo os mesmos
passos e continue falhando, se sentindo mais e mais frustrado. Talvez você saia para um
passeio, tome um café e decida voltar ao problema mais tarde. O que nenhum de nós do
mundo do Software Livre quer que aconteça é, na verdade, o que acontece para muitos:
aquela xícara de café é infinitamente melhor do que se sentir ignorante e intimidado,
e então você nunca mais irá colaborar em nenhum projeto de Software Livre.

Perceba que, um dia, você saberá todas aquelas coisas que os experts ao seu redor
já esqueceram ou não dão importância porque parecem óbvias para eles. Toda pessoa
mais experiente que você cumpriu a mesma peregrinação que você está lidando agora
ao aprender como fazer as coisas que você está tentando fazer. Aqui estão algumas
dicas para tornar sua viagem mais fácil:

\paragraph*{Não espere demais para pedir ajuda} Ninguém quer ser uma praga ou
parecer ser ignorante. Dito isto, se você não consegue resolver o seu problema
depois de tentar por 15 minutos, é hora de pedir ajuda. Certifique-se que você
checou a documentação no \textit{website} do projeto de modo a pedir ajuda no
canal de IRC, forum ou lista de discussão corretos. Muitos projetos possuem
canais de comunicação \textit{online} especialmente voltados para novatos, então
fique de olho em palavras tais como mentor, \textit{newbie} e primeiros passos (\textit{getting started}).

\paragraph*{Explique o que você fez} O ponto principal não é simplesmente fazer perguntas,
mas fazer as perguntas certas. No início, você não necessariamente saberá quais são as
perguntas certas então, ao pedir ajuda, seja detalhista em relação a o que você está
tentando realizar, aos passos que você executou e ao problema que você encontrou.
Faça com que o seu potencial mentor, no canal de IRC ou lista de discussão do projeto,
saiba que você leu o manual incluindo links para a documentação que você leu. Se você
não encontrou qualquer documentação, uma menção educada sobre este fácil também é útil.


\paragraph*{Reconheça seu próprio valor} Como um novo colaborador, você se torna
algo inestimável não por conta do seu conhecimento, mas por conta da sua ignorância.
Ao iniciar o seu trabalho com Software Livre, não parece tão óbvio (pra você) que não
seja digno de menção. Tome nota dos problemas que você encontrou e como eles foram
solucionados e então use estas notas para atualizar a documentação do projeto ou
trabalhar com a comunidade no desenvolvimento de vídeos explicativos ou outros
materiais voltados para um problema em particular. Ao encontrar algo realmente
frustrante, perceba que você tem a chance espetacular de ajudar o próximo novato
a não encontrar as mesmas dificuldades.
