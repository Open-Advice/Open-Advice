\chapterwithauthor{Kévin Ottens}{Universidade e comunidade}

\authorbio{Kévin Ottens é hacker da comunidade KDE de longa data. Ele contribuiu
para o KDE Platform no geral, com ênfase em design de API e arquitetura de
frameworks. Graduado em 2007, ele tem doutorado em Ciência da Computação que
o levou a trabalhar particularmente em engenharia de ontologias e sistemas multi-agente.
O trabalho de Kévin na KDAB inclui desenvolver projetos de pesquisa em torno das tecnologias
KDE. Ele mora em Toulouse onde atua como professor durante meio período em sua antiga
universidade.}

\section*{Introdução}
Comunidades de Cultura Livre são dirigidas principalmente por esforços de voluntários. Além disso,
a maioria das pessoas que entram nessas comunidades fazem isso durante seu tempo na universidade.
Esse é mais ou menos o período certo de sua vida para embarcar nessas aventuras:
você está jovem, cheio de energia, curioso e, provavelmente, quer mudar o mundo
à sua imagem. Isso é realmente tudo o que é preciso para a maioria dos trabalhos voluntários.

Mas, ao mesmo tempo, ser estudante não necessariamente deixa você com muito tempo 
para se engajar em uma comunidade de Cultura Livre. Na verdade, a maioria dessas comunidades
são enormes e pode ser assustador entrar em contato com elas.

Isso obviamente levanta uma questão assustadora: por que as comunidades de Cultura Livre
não tentam fazer divulgação nas universidades, atrair a próxima geração de colaboradores
talentosos? Essa é uma pergunta válida que tentamos explorar no contexto de uma comunidade
produtora de software, ou seja, a KDE. Nesse artigo, nós focamos nos aspectos que não previmos
mas com os quais tínhamos que lidar enquanto procurávamos uma resposta para essa questão.

\section*{Construindo uma relação com uma universidade local}
Na verdade, tudo começa quando se alcança os próprios estudantes, e para isso,
a melhor maneira ainda é chegar nas suas universidades, tentando mostrar a eles
quão acolhedoras as comunidades de Cultura Livre podem ser. Neste sentido, nós construímos
um relacionamento com a Universidade Paul Sabatier em Toulouse -- mais precisamente um de
seus cursos de estudos chamado IUP ISI que é focado em engenharia de software.

O IUP ISI era muito orientado para o conhecimento ``mão na massa'' e, por isso, tinha
um programa pré-existente para projetos de estudantes. Um ponto particularmente interesssante
desse programa é o fato desses estudantes trabalharem em equipes que misturam estudantes de
diferentes promoções. Estudantes de terceiro e quarto anos colaboram em um objetivo comum, 
que geralmente reúne equipes de sete a dez pessoas.

O primeiro ano do nosso experimento nos ligamos a esse programa, propondo novos
temas para os projetos, focando em software desenvolvido pela comunidade KDE.
Henri Massié, diretor do curso de estudos, acolheu muito bem a ideia e nos deixou
colocar o experimento no lugar. Para esse primeiro ano, nós alocamos duas vagas para
projetos de software relacionados ao KDE.

Para construir rapidamente a confiança, nós decidimos este ano fornecer algumas garantias
sobre o trabalho dos estudantes:
\begin{itemize}
  \item ajudar os professores a ter confiança nos temas abordados: os projetos
escolhidos eram próximos aos temas ensinados no IUP ISI (por isso que escolhemos
uma ferramenta de modelagem UML e uma ferramenta de gerenciamento de projetos para
esse ano);
  \item dar o máximo de visibilidade aos professores: nós fornecemos a eles um
servidor no qual a produção dos estudantes era regularmente compilada e acessível
remotamente para fins de teste.
  \item facilitar o engajamento dos estudantes com a comunidade: os mantenedores
dos projetos foram apontados para desempenhar um papel de ``cliente'' levando, assim,
exigências aos estudantes e ajudando-os a achar o seu caminho nas ramificações da
comunidade;
  \item finalmente, para os estudantes, nós introduzimos um mini-curso sobre como
desenvolver com Qt e os \textit{frameworks} produzidos pelo KDE;
\end{itemize}

No momento da redação desse artigo, já se passaram cinco anos desses projetos.
Pequenos ajustes para a organização foram feitos aqui e ali, mas a maioria das ideias
por trás deles permaneceram as mesmas. A maioria das mudanças feitas foram o resultado
de muito interesse da comunidade disposta a se engajar com os estudantes e de
muita liberdade dada a nós em temas que poderíamos abordar em nossos projetos.

Além disso, ao longo desses anos o diretor nos deu apoio e incentivo contínuos,
alocando efetivamente mais vagas para projetos da comunidade de Cultura Livre, 
provando que nossa estratégia de integração estava certa: construir a confiança de
forma rápida é a chave para um relacionamento entre uma comunidade de Cultura Livre
e uma universidade.

\section*{Compreendendo o ensino como um processo bidirecional}
Durante esses anos de construção de pontes entre a comunidade KDE e o curso de estudos
IUP ISI, nós acabmos em posições de ensino para apoiar os estudantes em suas tarefas
relacionadas ao projeto. Quando você nunca ensinou uma sala de aula cheia de estudantes,
você ainda pode ter essa imagem de si próprio sentado em uma sala de aula, há alguns anos.
De fato, a maioria dos professores já foram estudantes... às vezes nem mesmo o tipo
de estudantes disciplinados ou atentos. Você provavelmente tem essa sensação de beber
de uma mangueira de incêndio: um professor entrando em uma sala, ficando na frente dos
estudantes e entregando conhecimento a você.

Esse estereótipo é o que a maioria das pessoas tem em mente de seus anos como estudantes
e na primeira vez que entram numa situação de ensino elas querem reproduzir esse estereótipo:
vindo com o conhecimento para entregar.

A boa notícia é que nada pode estar mais longe da verdade do que esse estereótipo.
A má é que se você tenta reproduzi-lo, você provavelmente assustará seus alunos
e enfrentará nada mais que a falta de motivação por parte deles para se engajar com
a comunidade. A imagem que você passa de si mesmo é a primeira coisa que eles lembrarão
da comunidade: a primeira vez que você entra na sala de aula para eles \emph{você é} a
comunidade!

Não cair na armadilha desse estereótipo requer que você recue um pouco e compreenda
o que realmente é o ensino. Não é um processo de direção única onde alguém entrega
conhecimento aos alunos. Nós chegamos à conclusão de que, em vez disso, é um processo
bidirecional: você tem que criar uma relação simbiótica com seu aluno.
Tanto os alunos quanto o professor tem que deixar a sala de aula com novos conhecimentos.
Claro que você tem que entregar seus conhecimentos -- mas para fazer isso de forma eficiente
você precisa se adaptar às referências dos alunos o tempo inteiro. É um trabalho de muita humildade.


Entender que fato gera muito poucas mudanças na forma como você ensina:
\begin{itemize}
  \item Você terá que entender a cultura dos seus alunos. Eles provavelmente têm
uma experiência muito diferente da sua e você terá que adaptar o seu discurso
à eles; por exemplo, os estudantes que nós treinamos são todos parte da chamada
geração Y que tem características bastante diferentes das gerações anteriores
a respeito de liderança, lealdade e confiança .
  \item Você terá que reavaliar suas próprias habilidades, já que você precisará
adaptar seu discurso à cultura deles. Você abordará seu próprio conhecimento de
ângulos muito diferentes do que você está acostumado, o que inevitavelmente leva
você a campos que pensava dominar.
  \item Finalmente, você terá que desenvolver habilidades em apresentar; ensinar
significa sair de sua zona de conforto para apresentar seu próprio conhecimento
enquanto o mantém divertido e interesssante para o seu público. Isso o tornará um
apresentador melhor.
\end{itemize}

Como tal, você se tornará um professor melhor. Também os seus objetivos de conseguir
alunos bem treinados e engajados na comunidade de Cultura Livre serão satisfeitos.

\section*{Conclusão}
No final do dia, porque você iria fazer todo esse esforço para construir uma confiança
com uma universidade e sair de sua zona de conforto ao melhorar as suas aulas?
Bem, isso na verdade se resume à questão inicial que tentamos responder:

\emph{As comunidades de Cultura Livre não conseguem atrair novos colaboradores de
universidades simplesmente por causa de sua falta de ação?}

Em nossa experiência a resposta é \emph{sim}. Através desses cinco anos de construção
de um relacionamento com o IUP ISI, nós mantivemos em média dois alunos por ano. 
Alguns deles desapareceram depois de um tempo, mas outros se tornaram colaboradores
muito ativos. Os outros ainda manter alguma nostalgia desse período de sua vida e
continuam defendendo mesmo que não contribuam diretamente. E agora temos uma equipe
local do KDE que conseguiu organizar de forma eficiente uma conferência de dois dias
para a nossa festa de lançamento mais recente.

Desses ex-alunos, nenhum deles teria se envolvido com o KDE sem esses projetos
universitários. Teríamos perdido completamente esses talentos. 
Felizmente, não temos sido inativos.