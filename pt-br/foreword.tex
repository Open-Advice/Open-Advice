\section*{Prefácio}

Este é um livro sobre comunidades e tecnologia. É um livro que
representa um esforço coletivo, assim como as tecnologias que
construimos em conjunto. E se este é o seu primeiro contato com
a comunidade de Software Livre, você pode achar estranho pensar
em comunidades como a força motriz por trás da tecnologia. Tecnologias
não são criadas por grandes corporações? Na verdade, para nós é
quase o contrário.

Os autores deste livro são todos membros do que você poderia chamar
de "comunidade da liberdade do software". Um grupo de pessoas que
compartilham a experiência fundamental de que o software é mais
empoderador, mais útil, mais flexível, mais controlável, mais justo,
mais abrangente, mais sustentável, mais eficiente, mais seguro e --
em última análise -- simplesmente melhor quando ele apresenta quatro
liberdades fundamentais: de uso, de estudo/investigação, de compartilhamento
e de melhoria.

E enquanto há agora um número crescente de comunidades que deixaram para
trás a exigência de proximidade geográfica ao adotar tecnologias para
comunicação virtual, a comunidade de Software Livre foi pioneira neste
aspecto.

De fato, a Internet e a Communidade de Software Livre (Free Software)
\footnote{Para mim, Software de Código Aberto (Open Source) é um aspecto
desta comunidade. Este aspecto particular se articulou em 1998, algum
tempo após o surgimento da Internet. Por favor, fique à vontade para
substituir Software Livre por Software de Código Aberto se esta for a
sua terminologia preferida.} foram desenvolvimentos co-dependentes. À
medida em que a Internet cresceu, nossa comunidade pôde crescer com ela,
porém sem os valores e tecnologias da nossa comunidade, eu não tenho
dúvidas que a Internet não teria se tornado esta rede tão abrangente
que atualmente conecta pessoas e grupos ao redor do mundo.

Até hoje, nosso software é executado na maior parte da Internet, você
deve conhecer pelo menos alguns deles, tais como o Mozilla Firefox,
OpenOffice.org,\footnote{N.T. O projeto OpenOffice.org passou por uma ramificação
em 2010 e deu origem a dois projetos diferentes, o LibreOffice, mantido pela
The Document Foundation, e o OpenOffice.org, mantido pela Apache Software Foundation.}
Linux, ou talvez até mesmo o GNOME ou KDE. Mas nossa
tecnologia pode também estar oculta na sua TV, no seu roteador sem fio,
no seu caixa automático de banco, até mesmo no seu rádio, em sistemas de
segurança ou navios de batalha. O Software Livre está literalmente em
todo lugar.

O Software Livre foi essencial na criação de algumas das maiores
corporações que você conhece, como o Google, Facebook, Twitter e outros.
Nenhuma dessas empresas teria alcançado tanto em tão pouco tempo se não
fosse pelo poder que o Software Livre deu a elas de subir nos ombros daqueles
que vieram antes.

Mas existem muitas companhias menores cujo negócio é totalmente
voltado para Software Livre, incluindo a minha própria: Kolab Systems.
Uma co-participação ativa na comunidade, de forma íntegra e permanente,
se tornou um fator crítico de sucesso para todos nós. E isto vale até
mesmo para as grandes empresas, conforme a Oracle tem involuntariamente
demonstrado durante e após a aquisição da Sun Microsystems.

Porém, é importante compreender que nossa comunidade \textbf{não é}
anticomercial. Nós gostamos do nosso trabalho, e muitos de nós fizeram
disso sua profissão e principal fonte de renda. Dessa forma, quando
falamos em comunidade, falamos em estudantes, empresários, desenvolvedores,
artistas, escritores de documentação, professores, entusiastas, executivos,
vendedores, voluntários e usuários.

Sim, usuários. Mesmo que você não tenha percebido ou "nunca tenha se
associado a alguma comunidade," você de fato já é \emph{quase} parte
da nossa. A questão é se você irá decidir participar ativamente.

E isto é o que nos diferencia das gigantes monoculturas, das comunidades
isoladas, dos jardins murados de empresas como a Apple, Microsoft e
outras. Nossas portas estão sempre abertas. Nossas dicas, conselhos
e sugestões estão sempre disponíveis. O seu potencial também. Não há
limites para até onde você pode chegar -- depende meramente da sua
escolha pessoal, como foi para cada um de nós.

Então, se você ainda não faz parte da nossa comunidade ou está curioso
sobre o assunto, este livro representa um bom ponto de partida. Se
você já é um participante ativo, este livro pode ajudar na 
compreensão de uma série de facetas e perspectivas que podem ser novas
para você.

Pelo fato de este livro conter relatos importantes sobre aqueles
conhecimentos implícitos que geralmente construimos e transferimos
dentro das nossas subcomunidades, ele acaba funcionando em diferentes
tecnologias. Este conhecimento é tipicamente passado dos colaboradores
mais experientes para os novatos, razão pela qual é considerado bastante
óbvio e natural por aqueles já inseridos na comunidade.

Todo este conhecimento e cultura sobre como organizar trabalho colaborativo
nos permite construir tecnologia de qualidade com equipes pequenas e
distribuídas ao redor do mundo, superando barreiras de idiomas e culturas;
e com desempenho melhor que as grandes equipes de desenvolvimento das
maiores empresas do mundo.

Todas as pessoas que contribuíram na escrita deste livro são colaboradores
experientes em uma, ou as vezes muitas, áreas. Eles passaram a ser professores
e mentores. Ao longo dos últimos 15 anos, eu tive o prazer de conhecer a
maioria deles, trabalhar com muitos, e o privilégio de chamar alguns de
amigo.

Como Kévin Ottens acertadamente disse durante o Desktop Summit 2011 em
Berlim: "Construção de comunidades é construção de família e de amizade".

Portanto, é de fato com um profundo senso de gratidão que eu posso dizer
que não existe outra comunidade da qual eu faria parte, e eu espero
poder encontrar você em uma das conferências que estão por vir.
\newline
\begin{flushright}--- Georg Greve\end{flushright}
\begin{flushright}Zurique, Suíça; 20 de Agosto de 2011\end{flushright}

\textit{Georg Greve fundou a Free Software Foundation Europe (FSFE) em
2000 and foi seu presidente até 2009. Neste período, ele foi responsável
por projetar e construir muitas das atividades da FSFE, tais como o
programa para associados, fundamentos legais e políticos, e trabalhou
intensivamente com muitas comunidades. Atualmente, ele continua seu
trabalho como acionista e CEO da Kolab Systems AG, uma empresa totalmente
centrada em Software Livre. Pelas suas realizações ligadas
à Software Livre e padrões abertos, Geor Greve foi premiado com o
Federal Cross of Merit pela República Federal da Alemanha em 2009.}

\newpage
