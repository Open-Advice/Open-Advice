\chapterwithauthor{Felipe Ortega}{Prepare-se para o futuro: evolução de equipes em Software Livre}

\authorbio{Felipe Ortega é um pesquisador e gerente de projeto no Libresoft, um grupo
de pesquisa na Universidade Rey Juan Carlos (URJC), Espanha. Felipe desenvolve novas
metodologias para analisar comunidades colaborativas abertas (como projetos de Software
Livre, Wikipedia e redes sociais). Ele fez uma extensa pesquisa com o projeto
Wikipedia e sua comunidade de autores. Ele participa ativamente na pesquisa,
promoção e educação/formação em Software Livre, especialmente no Mestrado
em Software Livre na URJC. É um defensor ferrenho dos recursos educacionais abertos,
acesso aberto em publicações científicas e dados abertos na ciência.}

\noindent{}Em seu famoso ensaio \textit{A catedral e o bazar}\footnote{\url{
http://www.catb.org/~esr/writings/cathedral-bazaar/cathedral-bazaar}}, Eric S.
Raymond menciona uma das primeiras lições importantes que todo programador 
deve aprender: ``Todo bom trabalho de software começa por motivações pessoais 
do desenvolvedor''. Você nunca percebe quão correta é essa afirmação até você
enfrentar essa situação. Na verdade, a maioria dos programadores de Software Livre
(senão todos) certamente passaram por esse processo na medida em que eles começaram
a trabalhar em um novo projeto, ou se juntaram a um já existente, ansiosos por ajudar
a torná-lo melhor. Entretanto, muitos desenvolvedores e outros participantes em 
comunidades de Software Livre (escritores de documentação, tradutores, etc.) geralmente
negligenciam outra importante lição enfatizada por Raymond em seu ensaio: ``Quando você perde
interesse em um programa, sua última obrigação é entregá-lo a um sucessor competente''.
Esse é o tema central que eu quero abordar aqui. Você deveria pensar sobre o futuro
de seu projeto, e os recém-chegados que um dia assumirão o seu trabalho e continuarão
a melhorá-lo.

\section*{Alternância de gerações}

Em algum momento de suas vidas, muitos projetos de Software Livre vão enfrentar
uma alternância de gerações. Os antigos desenvolvedores responsáveis pela manutenção
e melhoria do código eventualmente deixarão o projeto e sua comunidade, por várias razões.
Estas incluem questões pessoais, um novo emprego que não lhes deixa tempo livre, um novo projeto,
a mudança para um projeto diferente que parece mais atraente, \dots\ A lista pode ser muito longa.

O estudo da alternância de gerações (ou rotatividade do desenvolvedor) em projetos de Software Livre
ainda é uma área de estudos emergente que requer mais pesquisas para melhorar nossa
compreensão dessas situações. Apesar disso, alguns pesquisadores já coletaram evidências
objetivas que lança alguma luz sobre esses processos.
No OSS 2006, meus colegas Jesus G. Barahona e Gregorio Robles apresentaram um trabalho
intitulado ``Contributor Turnover in Libre Software Projects''. Nesse trabalho,
eles mostram uma metodologia para identificar os desenvolvedores mais ativos (geralmente conhecidos
como \textit {core contributors}) em diferentes intervalos de tempo, ao longo de toda a história
de um determinado projeto. Então, eles aplicam esse método para estudar 21 grandes projetos, em
particular GIMP, Mozilla (antiga instância do famoso navegador) e
Evolution. Em poucas palavras, o que eles descobriram é que nós podemos identificar três tipos de
projetos de acordo com sua taxa de rotatividade de desenvolvedores:
\begin{itemize}
 \item Code gods projects (NÃO TRADUZI PQ NÃO FAÇO IDEIA DE COMO TRADUZIR ISSO)
: Esses projetos dependem fortemente do trabalho de seus 
fundadores e há muito pouca alternância de gerações, ou nenhuma. O GIMP se
enquadra nessa categoria.
 \item Projetos com várias gerações: Projetos como Mozilla mostram um claro
padrão de rotatividade de desenvolvedores, com novos grupos de desenvolvedores
ativos assumindo a liderança da manutenção e desenvolvimento do código das mãos
de \textit {core contributors} anteriores.
 \item Projetos compostos: Evolution pertence a uma terceira categoria de projetos,
mostrando alguma taxa de rotatividade mas não tão evidente como no caso anterior,
atenuado pela retenção de alguns \textit {core contributors} sobre a história do
projeto.
\end{itemize}

Essa classificação nos leva a uma pergunta óbvia: qual é o padrão mais
comum encontrado em projetos reais de Software Livre lá fora? Bem, os resultados
para o conjunto total de 21 projetos analisados neste trabalho rendem uma clara conclusão,
que é a de que várias gerações e projetos compostos são os casos mais comuns no ecossistema
do Software Livre. Somente o Gnumeric e o Mono mostraram um padrão distinto da forte retenção
de antigos desenvolvedores, indicando que as pessoam que contribuem para esses projetos podem 
ter mais razões mais apelativas para continuar seu trabalho por um longo tempo.

No entanto, essa não é a imagem normal. Pelo contrário, esse estudo fornece
suporte para o conselho que estamos considerando aqui, que deveríamos nos
preparar para transferir, em algum momento no futuro, nosso papel e conhecimento no
projeto para os futuros contribuidores que vão se juntar à nossa comunidade.

\section*{A lacuna do conhecimento}

Qualquer pessoa que experimenta uma mudança significativa em sua vida deve lidar
com a adaptação às novas condições. Por exemplo, quando você deixa o seu trabalho
por um outro você se prepara para um certo período no qual você tem que se ajustar
ao novo lugar e se integrar num grupo de trabalho diferente. Felizmente, depois de
um tempo você finalmente se estabelece em seu novo emprego. Mas, às vezes, você
mantem bons amigos de seu antigo emprego e você pode encontrá-los novamente após
a mudança. Talvez, então, falar com seus ex-colegas de trabalho, você pode saber
o que aconteceu com a pessoa recrutada para substituí-lo. Isso raramente ocorre em
projetos de Software Livre.

A desvantagem da alternância de gerações em projetos de Software Livre pode
vir de uma forma muito concreta, ou seja, uma lacuna de conhecimento. Quando uma
antiga desenvolvedora deixa o projeto e, especialmente, se ela tem uma vasta experiência
nessa comunidade, ela deixa para trás tanto o seu conhecimento tangível quanto o abstrato
que pode ou não ser passado aos novatos subsequentes.

Um claro exemplo é o código fonte. Como qualquer produto de fino trabalho intelectual
(bem, ao menos deve-se esperar isso, certo?) os desenvolvedores deixam uma marca
pessoal sempre que eles produzem um novo código. Algumas vezes, você se sente eternamente
em dívida com o incrível programador que escreveu o código limpo e elegante, que praticamente
fala por si só e é fácil de manter. Outras vezes, a situação é o oposto e você luta
para entender o tão obscuro e não claro código sem qualquer comentário ou dicas
que podem ajudá-lo.

Isso é o que tentamos medir em 2009, em um trabalho de pesquisa apresentado no HICSS
2009. O título é ``Using Software Archeology to Measure Knowledge Loss in
Software Projects Due to Developer Turnover''. Caso você esteja pensando, isso não
tem nada a ver com um chicote, tesouros, templos ou aventuras emocionantes, embora
tenha sido realmente divertido. O que medimos (entre outras coisas) foi a porcentagem
de código orfão deixado para trás por desenvolvedores que sairam de projetos de Software
Livre e não foram assumidos por nenhum dos desenvolvedores atuais, ainda. Nesse caso,
escolhemos quatro projetos (Evolution, GIMP, Evince e Nautilus) para testar nosso método
de pesquisa. E encontramos resultados muito interessantes.

O Evolution mostrou um padrão um pouco preocupante, no sentido de que a porcentagem
dos códigos orfãos era crescente ao longo do tempo. Em 2006, quase 80\% de todas as
as linhas de código tinham sido abandonadas pelos antigos desenvolvedores e permaneciam
intocadas pelo resto da equipe. Por outro lado, o GIMP mostrou um padrão radicalmente
diferente, com um esforço claro e sustentado da equipe de desenvolvimento para reduzir o
númerod de linhas de código orfãs. Aliás, lembre-se que o GIMP já tinha sido caracterizado
como um \textit {code gods project} e, assim, se beneficia de uma equipe de desenvolvimento
muito mais estável para realizar essa tarefa assustadora.

Isso significa que os desenvolvedores do GIMP tiveram uma melhor experiência que
as pessoas do Evolution? Para ser honesto, não sabemos. No entanto, podemos prever um
claro risco previsível: quanto maior a porcentagem de código orfão, maior o esforço para
manter o projeto. Sempre que você precisar corrigir um \textit {bug}, desenvolver uma
nova funcionalidade ou ampliar uma já existente, você topa com um código que nunca tinha
visto antes. Claro que você pode ser um programador fantástico, mas não importa quão maravilhoso
você seja, os desenvolvedores do GIMP têm uma clara vantagem neste caso, já que eles tem alguém
na equipe com o conhecimento preciso sobre a maior parte do código que eles precisam para manter.
Além disso, eles também trabalham para reduzir ainda mais a parte desconhecida do código fonte.

\section*{Sentir-se em casa}

Curiosamente, alguns projetos conseguem reter usuários por períodos muito maiores
que o esperado. Mais uma vez, podemos encontrar evidência empírica que apoiam essa 
afirmação. No OSS 2005, Michlmayr, Robles e González-Barahona apresentaram alguns
resultados relevantes sobre este aspecto. Eles estudaram a persistência da participação
de mantenedores de software no Debian, calculando a chamada taxa de meia-vida. 
Esse é o tempo necessário para uma determinada população de mantenedores caia pela
metade do seu tamanho inicial. O resultado foi que a meia-vida estimada dos mantenedores
do Debian era de aproximadamente 7.5 anos. Em outras palavras, uma vez que o estudo
foi realizado durante um período de seis anos e meio (entre Julho de 1998 e
Dezembro de 2004), que vai do Debian 2.0 ao Debian 3.1 (somente versões estáveis),
mais de 50\% dos mantenedores do Debian 2.0 ainda estavam contribuindo para o Debian
3.1.

O Debian criou um procedimento bastante formal para admitir novos mantenedores de
software (também conhecidos como desenvolvedores Debian) incluindo a aceitação do
Contrato Social do Debian e a demonstração de familiaridade com a Política do Debian.
Como resultado, seria de se esperar contribuidores bastante comprometidos. Na verdade,
esse é o caso, já que esses autores descobriram que pacotes abandonados por antigos
mantenedores era geralmente assumidos por outros desenvolvedores que permaneciam na
comunidade. Apenas nos casos em que o pacote não era mais útil ele era simplesmente
abandonado. Acho que podemos aprender algumas lições úteis desse trabalho de pesquisa:
\begin{enumerate}
 \item Gaste algum tempo desenvolvendo as diretrizes principais do seu projeto. Elas
 podem começar como um único e curto documento, simplesmente apresentando algumas
 recomendações e boas práticas. Isso deve evoluir à medida em que o projeto cresce,
 para servir como uma pílula de aprendizagem para os recém-chegados captarem rapidamente
 os valores principais de sua equipe, bem como a marca de seu estilo de trabalho.
 \item Obrigue-se a seguir padrões de código conhecidos, boas práticas e
estilo elegante. Documente seu código. Inclua comentários para descrever seções que
podem ser especialmente difíceis de entender. Não sinta que você está desperdiçando
seu tempo. Obrigue-se a seguir padrões de codificação conhecido, boas práticas e 
estilo elegante. Documente seu código. Incluir comentários para descrever seções
que pode ser especialmente difícil de entender. Não sinta que você está desperdiçando
seu tempo.Na prática, você está sendo muito pragmático, investindo tempo no futuro
do seu projeto.
 \item Se possível, quando chegar a hora de você sair do projeto tente avisar aos
 outros de sua decisão com alguma antecedência. Certifique-se de que eles entendam
 que partes críticas vão precisar de um novo mantenedor. Idealmente, se você estiver
 em uma comunidade, prepare pelo menos um procedimento muito simples para automatizar
 esse processo e se certifique-se de que você não esqueceu nenhum ponto importante
 antes que a pessoa deixe o projeto (especialmente se ela era um desenvolvedor importante).
 \item Fique de olho no tamanho do código órfão. Se ele cresce muito rapidamente, ou se
 atinge uma parte significativa do seu projeto, é uma clara indicação de que você terá
 problemas muito em breve, especialmente se o número de relatórios de bugs cresce ou
 você pretende renovar o seu código com um refatoramento sério.
 \item Certifique-se sempre de deixar dicas e sugestões suficientes para um recém-chegado
 que irá assumir o seu trabalho no futuro.
\end{enumerate}

\section*{Eu gostaria de ter sabido que você vinha (antes de eu sair)}

Eu admito que não é muito fácil pensar sobre seus sucessores enquanto você está programando. 
Muitas vezes, você simplesmente não percebe que o seu código pode acabar sendo tomado por um
outro projeto, reutilizado por outras pessoas ou você pode, eventualmente, ser substituído
por outra pessoa disposta a continuar o seu trabalho. No entanto, o bem mais precioso do
Software Livre é precisamente esse: o código será reutilizado, adaptado, integrado ou expandido
por outra pessoa. Sustentabilidade é uma característica crítica da engenharia de software.
Mas ela se torna primordial no Software Livre. Não é apenas em relação ao código fonte. 
É sobre as pessoas, relações sociais e etiqueta digital. É algo além do simples bom gosto.
Quod severis metes (``você colhe o que plantou''). Lembre-se disso, na próxima vez, você
pode ser o recém-chegado preenchendo a lacuna de conhecimento deixada por um antigo desenvolvedor.