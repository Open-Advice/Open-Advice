\chapterwithauthor{Markus Kr\"otzsch}{Fora do Laboratório: Construindo Comunidades Open Source a partir de Projetos Acadêmicos}

\authorbio{Markus Kr\"otzsch é um pesquisador de pós-doutorado no Departamento de
Ciência da Computação da Universidade de Oxford. Ele obteve seu Ph.D. no Instituto
de Informática Aplicada e Métodos Formais de Descrição (AIFB) do Karlsruhe Institute
of Technology (KIT) em 2010. Sua área de pesquisa é processamento automático e inteligente
de informação, variando desde fundamentos para representação formal de conhecimento até
aplicações tais como Semantic Web. Ele é o desenvolvedor-líder do Semantic MediaWiki
(plataforma de sucesso para aplicações de Semantic Web), co-editor da especificação
W3C OWL 2, mantenedor-chefe do portal semanticweb.org e co-autor do livro Foundations
of Semantic Web Technologies.}

\noindent{}Pesquisadores desenvolvem uma imensa quantidade de software na academia,
seja para validar uma hipótese, para ilustrar uma nova abordagem ou meramente como uma
ferramenta para auxiliar algum estudo. Na maioria dos casos, um pequeno protótipo pontual
já é suficiente e é rapidamente dispensado depois que o foco da pesquisa muda para
outra coisa. Entretanto, de vez em quando, uma nova abordagem ou tecnologia carrega
o potencial de realmente mudar a forma com que um problema é solucionado. Isso traz
reputação profissional, sucesso comercial e a gratificação pessoal de perceber todo
o potencial que uma nova ideia pode ter. O pesquisador que fez tal descoberta sente
então a vontade de transformar o seu protótipo em um produto que seja realmente utilizado
-- e se depara com um conjunto completamente novo de problemas práticos.

\section*{O Medo do Usuário -- N.T.: Ambíguo: quem tem medo? O dono do software ou o usuário? Não
sei como resolver :( Usuário tem que ser objeto e não sujeito (sentido de posse)}

Frederick P.\ Brooks, Jr., em um dos seus famosos ensaios sobre engenharia de software,
traça um excelente retrato sobre os esforços de manutenção de um software real e
nos adverte em relação aos usuários:

%
\begin{quote}
"O custo total de manutenção de um programa amplamente utilizado é, tipicamente,
no mínimo 40\% do custo do seu desenvolvimento. Surpreendentemente, este custo é
fortemente afetado pelo número de usuários. Mais usuários encontram mais \textit{bugs}."\footnote{
Frederick P.\ Brooks, Jr.: The Mythical Man-Month. Essays on Software Engineering. Anniversary
Edition. Addison-Wesley, 1995.}
\end{quote}
%
Embora os custos acima possam ser diferentes nos ambientes atuais, a observação básica
ainda é verdadeira e pode ainda ter sido agravada pelo uso de comunicação global
instantânea. Ainda pior, muitos usuários não somente encontram mais \textit{bugs} reais
como também articulam mais desejos em geral. Seja ela um erro genuíno, uma requisição de
funcionalidade ou meramente um mal entendido sobre a operação do software, uma requisição
típica de um usuário está longe de ser um \textit{bug report} tecnicamente preciso. Cada
requisição demanda a atenção dos desenvolvedores, consumindo um tempo precioso que passa a
não estar disponível para realmente escrever código.

A mente analítica do pesquisador prevê este problema e, em sua luta natural para prevenir
um futuro sombrio ao lidar com clientes, pode desenvolver um completo "medo do usuário".
No pior caso, isto pode levar a uma decisão contra o projeto como um todo. Em uma forma
menos severa, pode fazer com que pesquisadores escondam produtos brilhantes de software
dos seus potenciais usuários. Por mais de uma vez eu ouvi pesquisadores dizendo: "nós não
precisamos de mais visibilidade, já estamos recebendo e-mails demais!". De fato, há casos
onde o esforço de comunicação ligado a uma ferramenta de software excede o esforço que o
pesquisador pode investir sem sacrificar seu trabalho principal.

Na maioria dos casos, entretanto, este resultado trágico poderia ter sido facilmente evitado.
Brooks dificilmente poderia prever isso. Quando ele escreveu seu ensaio, usuário era de fato
clientes e manutenção do software era parte do produto que eles compraram. Era necessário
encontrar um equilíbrio entre esforço de desenvolvimento, demandas de mercado e preço.
Este é ainda o caso para muitos produtos comerciais de software atualmente, mas tem pouco a
ver com a realidade do desenvolvimento Open Source em pequena escala. Usuários típicos de
Software Livre não pagam pelo serviço que eles recebem. Consequentemente, suas atitudes não
são aquelas de um cliente exigente, mas sim as de um apoiador agradecido e entusiasta.
Um trabalho importante que os projetos de Software Livre de sucesso fazem é transformar
este entusiasmo no tão desejado suporte, fazendo com que o aumento no interesse dos usuários
se transforme também em um aumento na contribuição dos usuários.

Reconhecer que usuários de Software Livre não são simplesmente "clientes que não pagam" é
um discernimento importante. Porém, ele não deve nos levar a superestimar o potencial destes
usuários. A contrapartida otimista do medo irracional do usuário é a crença que comunidades
de Software Livre ativas e acolhedoras crescem naturalmente, baseado meramente na licença
escolhida para publicar o código. Este erro grave de julgamento ainda é surpreendentemente
comum e tem selado o destino de muitas tentativas de criação de comunidades de Software Livre.

\section*{Semeando e Colhendo}

O plural de "usuário" não é "comunidade". Enquanto o primeiro pode crescer em números,
o segundo não cresce por si só ou então cresce descontroladamente sem produzir o tão
sonhado suporte ao projeto. O trabalho do mantenedor do projeto -- que busca se beneficiar
da energia bruta dos usuários -- se parece, portanto, com aquele de um jardineiro que precisa
preparar um solo fértil, plantar e molhar as mudas de plantas e possivelmente podar aqueles
brotos indesejados antes de ser capaz de colher os frutos. Comparado com as recompensas, o
esforço geral é pequeno porém é vital fazer as coisas certas, no tempo certo.

\paragraph*{Preparando os Fundamentos Técnicos}
A construção de uma comunidade começa antes mesmo da chegada do primeiro usuário. A escolha
da linguagem de programação já determina quantas pessoas serão capazes de implantar e realizar
o \textit{debug} do código. Objective Caml pode ser uma linguagem bonita mas adotar o Java
irá aumentar, em muitas ordens de magnitude, o número de potenciais usuários e colaboradores.
Os desenvolvedores, portanto, devem chegar a uma solução de compromisso, visto que as tecnologias
mais amplamente adotadas raramente são as mais eficientes ou elegantes. Isto pode ser um passo
particularmente difícil para aqueles pesquisadores que frequentemente preferem aquelas linguagens
com design superior. Quando estava trabalhando no Semantic MediaWiki, frequentemente me perguntavam
porque nós usávamos PHP se Java \textit{server-side} seria muito mais simples e eficiente.
Comparar o tamanho da comunidade do Semantic MediaWiki com esforços similares baseados no Java pode
responder esta questão. Este exemplo também ilustra que o público-alvo determina a melhor escolha
das tecnologias base. O próprio desenvolvedor deve ter o discernimento necessário para tomar uma
decisão oportunista.

\paragraph*{Trabalhando os Fundamentos por Completo}
Um problema relacionado é a criação, desde o início, de código legível e bem documentado.
Em um ambiente acadêmico, alguns projetos de software são desenvolvidos por muitos colaboradores
temporários. Equipes transitórias podem deteriorar a qualidade do código. Eu me lembro de um
pequeno projeto de software em Tu Dresden que foi muito bem mantido por um aluno. Após a sua saída,
percebeu-se que o código estava amplamente documentado, em Turco. Um pesquisador será, no máximo,
um programador em tempo parcial. Dessa forma, é necessário uma disciplina especial para garantir
que o código esteja sempre acessível a todos. A recompensa é uma maior probabilidade de
\textit{bug reports} mais esclarecedores, \textit{patches} mais úteis e, mais tarde, colaboradores
externos.

\paragraph*{Espalhando as Sementes da Comunidade}
Desenvolvedores inexperientes de Software Livre frequentemente pensam que
publicar seu código abertamente é um grande passo. Na verdade, ninguém mais
irá notar. Para atrair usuários e colaboradores é necessário divulgar o projeto.
A comunicação pública de qualquer projeto real deve, no mínimo, incluir o anúncio
de cada nova versão. Listas de discussão são provavelmente o melhor canal para isso.
Alguma habilidade social é necessária para encontrar o equilíbrio certo entre um
\textit{spam} irritante e um eufemismo tímido. Projetos que são motivados por uma
convicção honesta de que eles irão ajudar usuários a resolver problemas reais são
geralmente fáceis de anunciar de forma respeitável. Os usuários irão facilmente notar
a diferença entre publicidade desavergonhada e informação útil. Obviamente, anúncios
mais audazes devem aguardar até que o projeto esta pronto o suficiente. Isto inclui
não somente o código-fonte mas também a \textit{homepage} do projeto e sua
documentação básica de uso.

Ao longo da sua vida, o projeto deve ser mencionado em todos os lugares apropriados,
incluindo \textit{web sites} (a começar pela página do próprio projeto!), apresentações,
artigos científicos e discussões online. Geralmente as pessoas não valorizam o poder
que um \textit{link} simples pode ter ao conduzir um colaborador em potencial à sua
primeira visualização da \textit{homepage} do projeto. Pesquisadores não devem se
esquecer também de divulgar o software fora do seu círculo acadêmico imediato. Dificilmente
uma comunidade ativa será formada apenas por outros pesquisadores.

\paragraph*{Criando Espaços para Crescer}
Algo trivialmente fácil, porém frequentemente negligenciado, é a obrigação do
mantenedor do projeto em disponibilizar os espaços de comunicação necessários
para o crescimento da comunidade. Se um projeto não possui uma lista de discussão
própria, todas as requisições de suporte serão enviadas diretamente para o
mantenedor. Se não há sistema público para rastreamento de \textit{bugs}, \textit{bug reports}
serão mais raros e menos úteis. Sem uma \textit{wiki} de documentação aberta ao público,
o desenvolvedor será o único responsável por estender e refinar a documentação. Se
o repositório do sistema de controle de versão não está acessível, usuários não serão
capazes de obter o código-fonte mais recente antes de submeter os \textit{bug reports}.
Se este repositório é fechado, é praticamente impossível admitir colaboradores externos.
Toda esta infraestrutura é disponibilizada gratuitamente por diversos provedores de
serviços. Nem toda forma de comunicação pode ser desejada, entretanto. Por exemplo,
existem razões para manter o grupo de desenvolvedores fechado. Porém, seria insensato
esperar algum suporte da comunidade sem preparar os espaços básicos para isso acontecer.

\paragraph*{Encorajando e Controlando o Crescimento}
Desenvolvedores inexperientes frequentemente acham que disponibilizar listas de discussão,
foruns e \textit{wikis} para usuários requer muita manutenção adicional. Isto raramente
acontece mas algumas atividades básicas são realmente necessárias. Tudo começa com uma
solicitação rigorosa para que os espaços de comunicação sejam utilizados. Usuários precisam
ser educados para realizar questões de forma pública, consultar a documentação antes
de questionar e a submeter \textit{bug reports} via sistema de rastreamento em vez de e-mail.
Eu tenho uma tendência a rejeitar todas as requisições privadas de suporte, ou a repassar
as respostas para listas de discussão públicas. Isso também garante que soluções estarão
disponíveis na web para futuros usuários. Em qualquer caso, deve-se agradecer explicitamente
a todos os usuários por qualquer forma de contribuição -- muitas pessoas entusiasmadas e
bem-intencionadas são necessárias para construir uma comunidade saudável.

Quando se alcança uma certa densidade de usuários, o suporte começa a acontecer de
usuário para usuário. Este é sempre um momento mágico para o projeto e é um sinal
certo de que ele está seguindo um bom caminho. Idealmente, os mantenedores centrais
devem ainda ajudar nas questões mais complicadas mas, em algum momento, os usuários
irão liderar as discussões e é importante agradecê-los (diretamente) e envolvê-los
cada vez mais no projeto. De forma contrária, situações insalubres devem ser contornadas
sempre que possível e comportamento agressivo, em particular, pode ser um perigo real
para o desenvolvimento da comunidade. Da mesma forma, nem todo entusiasmo bem-intencionado
é produtivo. Neste momento, é frequentemente necessário dizer não, de forma amistosa porém
clara, para evitar \textit{feature creep}\footnote{N.T. Fenômeno no qual um software
implementa mais funcionalidades do que originalmente planejado, geralmente com redução
na qualidade das implementações.}.

\section*{O Futuro é Ilimitado}

Construir uma comunidade initial ao redor de um projeto é uma parte importante
da transformação de um protótipo de pesquisa em um projeto maduro de Software Livre.
Se isto acontecer com sucesso, muitas opções para ampliação do projeto estão disponíveis,
dependendo das metas do mantenedor e da comunidade. Algumas possibilidades gerais incluem:

%
\begin{itemize}
\item Continuar crescendo e desenvolvendo o projeto e a comunidade,
ampliando a equipe central de desenvolvedores/mantenedores e eventualmente tornando
o projeto independente das suas origens acadêmicas. Isto pode requerer mais
atividades na comunidade (ex: eventos dedicados) e talvez o estabelecimento de
suporte organizacional.
%
\item Criar uma empresa para explorar comercialmente o projeto baseado, por exemplo,
em licenças duais ou consultoria. Ferramentas de ampla aceitação no mercado e
comunidades vibrantes são ativos fundamentais de qualquer empresa e podem ser
benéficas para várias estratégias de negócio, sem abandonar o produto de Software
Livre original.
%
\item Retirar-se do projeto. Existem muitas razões pelas quais alguém pode não
mais ser capaz de manter uma relação de proximidade com o projeto. Ter criado
uma comunidade saudável maximiza as chances do projeto continuar de forma independente.
Em qualquer caso, é muito mais sensato comunicar claramente a retirada do que abandonar
o projeto silenciosamente, o destruindo por inatividade até o ponto em que nenhum
mantenedor futuro pode ser encontrado.
\end{itemize}
%
O formato da comunidade será diferente ao caminhar na direção de uma dessas possibilidades
gerais. Mas independente disso, o papel do pesquisador muda em função do projeto. O cientista
e programador inicial pode se tornar um gerente ou diretor técnico. Neste sentido, a principal
diferença entre um projeto de Software Livre influente e um protótipo perpétuo de pesquisa
não é a quantidade de trabalho, mas sim o tipo de trabalho necessário para se ter sucesso.
Compreender isto é parte do sucesso -- a única outra coisa necessária é um software incrível.
