\chapterwithauthor{Armijn Hemel}{Programe Primeiro}

\authorbio{Armijn Hemel usa Software Livre desde 1994, quando seu irmão
chegou em casa com uma pilha de disquetes com uma versão antiga do FreeBSD.
Um ano depois a transição para o Linux foi feita e ele vem usando sistemas
Unix-like desde então, tanto em casa quanto durante seus estudos na Uthrecht
University e no trabalho.
Desde 2005, Armijn é parte da equipe central do gpl-violations.org e possui
sua própria consultoria (Tjaldur Software Governance Solutions) especializada
na detecção e resolução de violações da licença GPL.}

\noindent{}Em 1999 eu estava apenas começando no ativismo do Software Livre.
Eu já vinha usando o Linux e o FreeBSD há alguns anos naquela época, mas
eu era meramente um usuário e eu queria realmente passar a contribuir com
alguma coisa. Eu pensei que a melhor forma de começar a contribuir seria escrevendo
código. Eu não pude encontrar um projeto já existente no qual eu me sentisse
confortável para trabalhar, então eu decidi começar meu próprio projeto. Pensando
em retrospectiva, a razão pela qual eu fiz isso foi provavelmente uma mistura de
várias coisas. Uma delas foi a insegurança sobre se o meu código era bom o
suficiente para ser aceito em um projeto já existente (eu não era, e ainda não
sou, um programador brilhante) e, com o seu próprio projeto, isso não era um
problema sério. A segunda razão foi provavelmente arrogância juvenil.

Minha ideia era construir um programa para apresentações multimídia, que possuísse
aquelas funcionalidades avançadas (ou chatas, se você preferir) do PowerPoint.
Naquela época não havia o OpenOffice.org e as alternativas eram praticamente
limitadas ao LaTeX e Magicpoint, as quais eram mais direcionadas à produção de
conteúdo textual do que à exibição de efeitos hipnóticos. Eu queria que o programa
fosse multiplataforma e então eu pensei que o Java seria a melhor escolha para isso.
A ideia era construir um programa para apresentações, escrito em Java e que suportasse
todos aqueles efeitos hipnóticos. Eu tomei a decisão e iniciei o projeto.

Em relação à infraestrutura, tudo estava lá: havia uma lista de discussão,
um website e um sistema de controle de versão (CVS). Mas não havia código de verdade
para as pessoas trabalharem. As únicas coisas que eu tinha eram algumas ideias sobre o
que eu queria fazer, um desejo para satisfazer e os clichês certos. Eu realmente
queria que outras pessoas contribuissem na construção deste programa e o tornasse um
projeto realmente colaborativo.

Eu comecei a fazer o projeto do software (com algum conhecimento de UML recém adquirido)
e o tornei disponível. Nada aconteceu. Eu tentei fazer com que as pessoas se envolvessem,
mas trabalhar de forma colaborativa em projetos UML é bastante difícil (além disso,
provavelmente não é a melhor forma de iniciar um projeto de software). Depois de um
tempo eu desisti e o projeto morreu silenciosamente, sem ao menos produzir uma única
linha de código. Todo mês eu era lembrado, pelo software de lista de discussão, que
o projeto uma vez existiu e então eu pedi para que ele fosse oficialmente extinto.

Eu aprendi uma lição muito valiosa, mas de alguma forma dolorosa: quando você
anunciar algum projeto e quiser que pessoas se envolvam, certifique-se que há
ao menos algum código disponível. Não é necessário que esteja completamente finalizado,
ele pode estar bastante incipiente (é normal estar assim no início), mas ao menos
mostre que existe algo para as pessoas trabalharem e melhorarem. Caso contrário, seu
projeto terá o mesmo destino de muitos outros, incluindo o meu próprio: o esquecimento.

Eu finalmente encontrei minha forma de colaborar com o Software Livre: garantir, através
do projeto gpl-violations.org, que os fundamentos legais do Software Livre sejam rigorosos
o suficiente. Em retrospectiva, eu nunca usei e também nunca senti falta dos efeitos
hipnóticos dos programas de apresentação, percebi que eles são irritantes e que distraem
a platéia do conteúdo real. Sou um feliz usuário do LaTeX Beamer e ocasionalmente (com
uma dose menor de felicidade) uso o OpenOffice.org/LibreOffice para criar apresentações.
