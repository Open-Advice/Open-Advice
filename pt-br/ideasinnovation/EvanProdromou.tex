\chapterwithauthor{Evan Prodromou}{Todos os outros podem estar errados, mas provavelmente não estão}

\authorbio{Evan Prodromou é o fundador da Wikitravel, StatusNet e da rede social 
de código aberto Identi.ca. Ele participa de comunidades de Software Livre há 15
anos como desenvolvedor, escritor de documentação e como um ocasional atirador 
de bombas. Ele mora em Montreal, Quebec.}

\noindent{}A característica mais importante de um fundador de projetos de Software Livre -- 
nas primeiras semanas ou meses antes de lançar seu software para o mundo -- é aquela 
inabalável persistência diante de provas contrárias esmagadoras. Se seu software é tão
importante por que uma outra pessoa ainda não o fez? Talvez ele não seja possível de ser construído. 
Talvez ninguém mais queira fazer o que você está fazendo. Talvez você não seja bom o suficiente
para desenvolvê-lo. Talvez uma outra pessoa já o tenha feito e você apenas não foi bom o
suficiente para encontrá-lo através de pesquisas na Internet.

Continuar persistindo por tanto tempo, diante de tantas incertezas, é difícil; apenas aqueles mais teimosos,
obstinados e persistentes conseguem. É necessário exercitar e defender com firmeza as suas opiniões 
mais técnicas. Qual é a melhor linguagem de programação a ser adotada?
Qual será a arquitetura da aplicação? Padronizações de código? Cores de ícones? Licenças de software? Qual sistema de
controle de versão adotar? Se você é o único colaborador do projeto, então você
terá que tomar estas decisões sozinho.

Entretanto, a partir do momento em que você disponibiliza a primeira versão do software, essa característica essencial de 
determinação teimosa e opinião forte se torna um malefício em vez de um benefício. Após o 
lançamento da aplicação, você precisará exatamente da habilidade oposta, ou seja, fazer concessões para
tornar seu software mais útil para outras pessoas. E muitas dessas concessões irão parecer
realmente erradas.

É difícil aceitar sugestões de "estranhos" (pessoas que não são você). 
Primeiro porque eles focam naquelas coisas triviais e sem importância -- a sua forma
de nomear variáveis ou a disposição de certos botões. Depois, porque eles
estão invariavelmente errados -- afinal, se a forma como você fez não é a forma correta,
você não teria feito assim desde o início. Se a sua solução não fosse a correta, por que
o seu código seria tão popular?

Mas o conceito de "errado" é relativo. Se fazer uma escolha "errada" torna seu software mais
acessível para usuários finais, desenvolvedores de distribuições ou administradores/empacotadores,
esta decisão não seria, na verdade, correta?

A natureza desses tipos de comentários e contribuições geralmente é negativa.
O feedback da comunidade é essencialmente reativo, ou seja, é primariamente crítico.
Quando foi a última vez que você criou um \textit{bug report} simplesmente para falar: "eu realmente gosto da
organização do módulo \texttt{hashtable.c}" ou "o \textit{design} deste menu ficou muito bom"? 
As pessoas dão feedback porque não gostam da forma como as coisas 
atualmente funcionam no seu software. Além disso, elas podem ser nada diplomáticas ao enviar
seus comentários.

É difícil responder de forma positiva a esse tipo de feedback. Às vezes, nós insultamos
tais pessoas em listas de discussão ou fechamos os \textit{bug reports} 
com desdém e status "WONTFIX"\footnote{Abreviação para a frase em inglês "I won't fix", que
significa "Eu não corrigirei".}. Ou pior, nos fechamos em nossos casulos, ignorando sugestões
ou feedbacks externos, cultivando aquele código confortável que se encaixa perfeitamente
em nossos preconceitos e vieses.

Se seu software é apenas para você, então você pode fazer do código e da
infra-estrutura adotada o seu playground pessoal. Mas se você deseja que seu software 
seja realmente utilizado, que signifique algo para outras pessoas e que (talvez) mude o mundo,
então você vai precisar construir uma comunidade próspera e orgânica,
formada por usuários, desenvolvedores e administradores. 
As pessoas precisam se apropriar do software e ter a mesma sensação de pertencimento que você tem.

É difícil lembrar que cada uma dessas vozes discordantes é um pequeno pedaço
de algo maior. Imagine todas as pessoas que ouviram falar sobre seu software mas
nunca se deram ao trabalho de usá-lo. Imagine aquelas que fizeram o download mas
nunca o instalaram. Aquelas que o instalaram, tiveram dificuldades e silenciosamente
desistiram. E aquelas que queriam dar um feedback mas não conseguiram
encontrar o sistema de \textit{bug report}, a lista de discussão dos desenvolvedores,
o canal do IRC ou os endereços de e-mails pessoais. Dadas as barreiras de comunicação,
para cada 1 pessoa que consegue enviar uma mensagem existem provavelmente cerca
de 100 outras pessoas com potenciais feedbacks. Dessa forma, ouvir aqueles que conseguiram
chegar até você é fundamental.

O líder do projeto é responsável por manter a visão e o propósito do software.
Nós não podemos vacilar, titubeando de um lado a outro a cada este ou aquele
e-mail que chega de um usuário qualquer. Se há um princípio fundamental em jogo, então
é naturalmente importante mantê-lo forte e estável. Somente o líder do projeto
é capaz de fazer isso.

Porém, é necessário pensar: existem questões não-fundamentais que podem tornar o seu
software mais acessível ou útil? Em última análise, a medida do seu trabalho
está na forma como você alcança as pessoas, na forma como seu software é utilizado e para
que ele é utilizado. Quanto da sua ideia pessoal sobre o que é "certo" realmente
importa para o projeto e para a comunidade? Quanto disso representa simplesmente o que você
pessoalmente gosta? Se essas questões não-fundamentais existem, reduza os atritos,
responda às demandas e faça as mudanças. Isso fará com que o projeto seja melhor
para todos.
