\chapterwithauthor{Evan Prodromou}{Os outros podem estar errados, mas provavelmente não estão}

\authorbio{Evan Prodromou é o fundador da Wikitravel, StatusNet e da rede social 
de código aberto Identi.ca. Ele participa de comunidades de Software Livre há 15
anos como desenvolvedor, escritor de documentação, e como um ocasional atirador 
de bombas. Ele vive em Montreal, Quebec.}

\noindent{}A característica mais importante de um fundador de projeto de Software Livre, 
nas primeiras semanas ou meses antes de lançar seu software para o mundo, é a 
persistência teimosa diante de provas factuais esmagadoras. Se seu software é tão
importante, por que alguém ainda não o tinha escrito? Talvez isso não seja possível. 
Talvez ninguém mais deseje fazer o que você está fazendo. Talvez você não seja bom o suficiente
para fazer isso. Talvez alguém mais já fez, e você apenas não foi bom o suficiente para encontrar
pesquisando na Internet.

Manter a fé por tanto tempo, na noite escura, é difícil; apenas a pessoa mais teimosa,
obstinada e persistente consegue. E nós temos que pôr em prática todas as nossas opiniões 
de programadores defendidas com firmeza. Qual é a melhor linguagem de programação para usar?
Arquitetura da aplicação? Padrões de código? Cores do ícone? Licenças de software? Sistema de
controle de versão? Se você é o único a trabalhar no projeto (ou que sabe sobre ele!), você
tem que decidir sozinho.

Quando você finalmente lança seu software, no entanto, essa característica essencial de 
determinação teimosa e opinião forte se torna um malefício, não um benefício. Depois do 
lançamento, você precisará exatamente da habilidade oposta, ou seja, fazer concessões para
tornar seu software mais útil para outras pessoas. E muitas dessas concessões você sentirá
que são realmente erradas.

É difícil pegar informações de ``estranhos'' (por exemplo, pessoas que não são você). 
Primeiro, porque eles focam no trivial, coisas sem importância -- a sua forma
de nomear variáveis, digamos, ou a colocação de certos botões. E segundo, porque eles
estão invariavelmente errados -- afinal, se a forma como você tem feito não é a forma certa,
você não teria feito assim, em primeiro lugar. Se sua forma não era a forma certa, por que
seu código seria popular?

Mas ``errado'' é relativo. Se fazer uma escolha ``errada'' torna seu software mais
acessível para usuários finais, ou para os desenvolvedores das distribuições, ou para administradores
ou empacotadores, isso não está, na verdade, certo?

E a natureza desses tipos de comentários e contribuições geralmente é negativa.
O feedback da comunidade é essencialmente reativo, o que significa que é em primeiro lugar crítico.
Quando foi a última vez que você preencheu um relatório de \textit{bug} e disse: ``Eu realmente gosto da
organização do módulo hashtable.c.'' ou ``Bom trabalho em montar esse
sub-sub-sub-menu.''? As pessoas dão feedback porque não gostam da forma como as coisas 
funcionam agora com seu software. Elas também podem não ser diplomáticas ao fazer isso.

É difícil responder a esse tipo de feedback positivamente. Às vezes, nós incendiamos
nossas listas de discussão de desenvolvimento, ou fechamos relatórios de \textit{bug} 
com desdém e um WONTFIX\footnote{Abreviação para a frase em inglês "I won't fix", que
significa "Eu não corrigirei"}. Pior, nos fechamos em nossos casulos, ignorando sugestões
de fora ou feedback, abraçando nosso código confortável que se encaixa perfeitamente
em nossos preconceitos e vieses.

Se seu software é apenas para você, você pode manter a base do código e sua
infra-estrutura como um playground pessoal. Mas se você deseja que seu software 
seja usado, para significar algo para outras pessoas, para (talvez) mudar o mundo,
então você vai precisar construir uma comunidade de usuários próspera, orgânica,
com um núcleo de \textit{committers}, administradores e desenvolvedores extras. 
As pessoas precisam sentir como se possuíssem o software, da mesma forma que você.

É difícil lembrar que cada uma dessas vozes discordantes é o pequeno pedaço
de algo maior. Imagine todas as pessoas que ouvem sobre seu software e
nunca se deram ao trabalho de usá-lo. Aquelas que fizeram o download dele mas
nunca o instalaram. Aquelas que o instalaram, tiveram dificuldades, e silenciosamente
desistiram. E aquelas que queriam dar um feedback a você, mas não conseguiram
encontrar o sistema de relatório de \textit{bug}, a lista de e-mail de desenvolvedores,
o canal do IRC ou endereços de e-mail pessoais. Dadas as barreiras de comunicação,
existe provavelmente cerca de 100 pessoas que gostariam de ver uma mudança para 
cada pessoa que manda uma mensagem. Então, ouvir essas vozes quando elas
chegam até você, é crucial.

O líder do projeto é responsável por manter a visão e o propósito do software.
Nós não podemos vacilar, avançando ou retroagindo apenas baseado neste ou naquele
e-mail de usuários aleatórios. E se há um princípio fundamental em jogo, então,
naturalmente, é importante mantê-lo estável. E ninguém além do líder do projeto
pode fazer isso.

Mas, temos que pensar: existem questões não fundamentais que podem tornar o seu
software mais acessível ou útil? Em última análise, a medida do nosso trabalho
está na forma como alcançamos as pessoas, como nosso software é usado, e para
o que ele é usado. Quanto da nossa ideia pessoal sobre o que é ``certo'' realmente
importa para o projeto e para a comunidade? Quanto do projeto é apenas o que o líder
gosta, pessoalmente? Se essas questões não fundamentais existem, reduza o atrito,
responda as demandas, e faça as mudanças. Isso vai tornar o projeto melhor
para todos.
