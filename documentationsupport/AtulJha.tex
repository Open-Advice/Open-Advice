\chapter{Life Changer Documentation for Novice -- Atul Jha}
\todo{bio}
In year 2005, cyber cafe was the place to surf Internet as dial up connection was very costly. During that time, yahoo chat was very popular and i used to visit \#hackers channel there. There were some crazy people there who said they will hack my password. I was very eager to know more about hacking and become one of them. The next day i went to cyber cafe again and typed “how to become a hacker” on yahoo search. The very first url was of Eric S Raymond, i was jumping with joy with perception that i got the magic key.
 
I started reading the book and to my surprise the definition of hacer was “someone who likes solving problems and overcoming limits”. It also said “hackers build things, crackers break them.”  Alas i wanted to be cracker but this book bought me to the other world og hacking. I kept reading the book and enountered various new terms like Gnu/Linux, Mailing list, Linux user group, Irc, python and many more. 

After searching further more, i was able to find Linux user group in Delhi and got chance to meet real hackers. I felt like am in an alien world as they were talking about Perl, RMS, Kernel, device drivers, compilation and many other things which was going over my head.  

I was in different world and preffered coming back home and find some Linux Distribution from somewhere. I was too scared to ask for one from them, i was nowhere near them a total dumb newbie. I managed some distribution by paying 1000 Rs to a guy who used to have business of selling distribution. I tried many of them and was not able to get my sound working. This time i decided to visit a Irc channel from the cyber cafe. I found \#linux-india and jumped over asking “my sound nt wrking”, then came instructions like no SMS speak and RTFM. It scared me more and took sometime to figure out what RTFM meant. 

I was terrified and preffered staying away from IRC for few weeks. One fine day i got mail about monthly Linux-user group meetup. I needed answers for my many questions. I met Karunakar there and he asked me to bring my CPU to his office as he had whole Debian repository available there. Debian was new term for me but i was satisfied with the fact that finally i will be able to play music  on Linux.  Next day i was in his office carrying my CPU on the over crowded bus, it was fun. In few hours Debian was up and running on my system. He also gave me few beginners book and command refernce. 

Next day again in Cyber cafe i read ESR another piece “how to ask question in smart way”. I was still visiting \#hackers channel at yahoo chat where i made a very good friend Krish who suggested me to buy a book called “linux command refernces”.  After spending sometime on those books and looking up at tldp.org i was a newbie Linux user. I had no looking back after that. I also attended a Linux conference where i met few hackers from yahoo and i was really inspired after attending there talk. Later after few year i had chance to meet RMS who is more like god for many people in Free software community. 

I would admit that documentation of ESR changed my life and to many others for sure. After all this years in the Free Software community i have realized documentation is the key for participation of newbies to this awesome open source community. My 1\$ advice to all developers would be to please document smallest work you do as world is full of newbies who would love to read them. My blog has simplest posting like enabling spellchecker in open-office to installing Django in virtual enviornemnt.
