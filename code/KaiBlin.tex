\chapter{Writing Patches -- Kai Blin}

\textit{Kai Blin is a computational biologist searching for antibiotics in his
    day job, both at the computer and in the lab. He feels very happy that he
    gets to release the software developed at work under Open Source licenses.
    Living in the lovely southern German town of T\"ubingen, Kai spends some of
    his evenings at the computer, programming for the Samba project. Most of
    his remaining spare time is spent at the theatre, where Kai is active both
    on the stage as well as building props, stage and handling other techie
    things backstage.}

Writing patches and submitting them often is the first real interaction you can
have with an Open Source project. They are the first impression you give to the
developers there. Getting your first patches ''right'', however that is judged
by the particular project you are contributing to, will make your life much
easier.

The exact rules on how the patch should look like, how you need to send it to
the project and all the other details will probably vary with every project you
want to contribute to. I have found that few general rules hold pretty much all
the time, and that is what this essay is about.

\section*{How to get things wrong}

Part 1 text


\section*{How NOT to get things wrong}

Part 2 text


\section*{Conclusion}

Conclusions
