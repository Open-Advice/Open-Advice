\chapter{Sally Khudairi}
\todo{needs title}
\textit{Active in the Web since 1993, Sally Khudairi is the publicist behind some of the industry's most prominent standards and organizations. The former deputy to Sir Tim Berners-Lee and long-time champion of collaborative innovation, she helped launch The Apache Software Foundation in 1999, and was elected its first female and non-technical member. Sally is Vice President of Marketing and Publicity for The Apache Software Foundation, and Chief Executive of luxury brand communications consultancy HALO Worldwide.}

\section*{Never care more than the client}
I am passionately committed to my work. A huge part of what I do involves strategizing and problem-solving on executive-level issues and working hands-on with tactical teams to ensure the job gets done correctly. I'm often up in the middle of the night trying to find ways to deal with a crisis or trying to finish an inordinate amount over work under extreme deadlines. I've long been openly willing to lay myself on the tracks with a speeding train approaching because I cared so much about a project. Minding the shop and round-the-clock firefighting whilst the client is on the golf course doesn't work, however. Something's amiss when you're burning the midnight oil long after everyone else has called it a day. Whether in-house or agency, if your client/boss/coworkers/colleagues refuse see the importance of a situation much less care about it, then change your perspective, put the project on a shelf, or learn to walk away. Loyalty is one thing, but blinding yourself to reality devalues your own efforts.

\section*{Never on a Friday}
The worst day to launch a new website, issue a press release, or brief the media is on a Friday. The chance that something wrong will happen with nobody available to deal with the fallout is greater than you can imagine. A poignant reminder of this happened to me early in my career when I launched the new W3C homepage on a Friday evening, left the office and boarded a plane for Paris. Coming from the world of commercial Web publishing, using a proprietary tag wasn't an issue whatsoever as long as it got the job done. Doing so on the Website of an interoperability-all-the-way organization on the other hand was Not A Good Thing. Within minutes dozens of messages were pouring in, wondering how the <now-deprecated-markup> tag got on our site. And no, it wasn't <blink>...

\section*{Never think that it doesn't matter}
Credibility is everything. Despite being overworked/overcommitted/overextended, you can't un-strike a bell. Try to deliver as much as you can to the best of your ability and ask for help if you can. Some deadlines have to be adjusted, and many editors can accommodate shift in schedule but it likely won't matter (as much) once the story/fire's gone out if you're unable to follow through. Like art, standards development, and copywriting, the process can go on ad nauseam. Whilst creativity can't be time-managed, hard deadlines force a line to be drawn at some point. But you've got to care about the details. Stop. Proof-read and check all links. Make sure it maps properly to the overall campaign/brand strategy. Lather-rinse-repeat is part of the greater communications gestalt, and the work will keep piling up. Sort it out and protect your reputation.

\section*{Do go at it alone}
It's important to trust your instincts, particularly when doing something separate from the norm. In the early days of that newfangled Web thaang, everyone was seemingly tacking on the usual branding/PR/marketing tactics to a brochure-ware Website. Then everyone was "following the leader" (= whoever did it first in many instances). Trends are one thing, industry expectations/requirements are another: "that's how everybody does it" doesn't mean that it's right for you, your project, or community. My career in communications began when I fired our retained agency and brought everything in-house. We were one of the earliest organizations to use a URL in a corporate boilerplate, and were the first to use a URL as the originating location on a press release dateline despite news wire agencies telling it was non-conformant and against policy. Stand confidently in your knowledge. Go against the grain and challenge the rules responsibly. Individuate. It's OK to be a dissenter as long as you can back your ideas up.

\section*{Do ease your expectations}
I'm known for pushing people and issues incredibly hard at times. During a table-pounding meltdown in Tim Berners-Lee's office, he said something that stopped me in my tracks: "...you don't seem to understand the meaning of 'good enough'." To me, everything I’m involved with matters: whilst I try to not sweat the small stuff, my standards remain high. As long as we're getting to the goal, how we get there may vary. I have to remind myself that not everyone will see things the same way, or have the same experience, or do things the way I would. The ramp-up time may be slower in some instances. Let others have a go at it; their creativity may surprise you. And if it fails, stet.

\section*{Do take care of yourself}
Coming from a field (art + architecture) where pulling all-nighters were par for the course, I didn't think anything of having a 16-hour workday. Ten hours of constant interruptions would be bookended with a bit of work with the Europeans early in the day and with those on the West Coast and Asia late at night. I was burning out fast couldn't see any other way of doing things. Yet old habits die hard: when I was on medical leave for six months, I hardly changed my ways. When asked if having cancer changed my lifestyle, the only thing that was significantly affected was my travel schedule. Three days after treatment, I moved to San Francisco, commuted 2+ hours and immersed myself in a 12-hour workday at a Silicon-Valley startup. I was exhausted, overworked, unrecovered from the side effects from treatment, and not yet healed. The HR and administrative teams maintained a 9-5-ish timeframe, and I was considered a lightweight. Within a year I left the corporate spin-cycle, moved back East, and had to figure out a better way to live. All-nighters still crop up from time to time. If I can't do something, I usually won't force myself. If they don't like it, too bad. As the airline safety guidelines remind us, if you can't manage to put on your own oxygen mask, you're not good for anybody.

\section*{Don't be surprised to take it from all sides}
Everyone has an opinion. And they'll likely give it to you.

\section*{Don't touch it for 24 hours}
Sometimes you need to walk away. From a project, from an argument, from work altogether. Give yourself a break and try to pace yourself; allow a day for things to settle down and for you to get a chance to breathe. Whilst that's usually not possible in a deadline-driven industry, it's something to aim for. The mad rush, non-stop emails, and continuous tweets often trigger reactions for emergencies that don't exist. Put the project down, clear your head, and come back with a fresh perspective. Step aside and regain your life.

\section*{Don't get sucked in the undertow}
The interconnectedness of all things proves that opportunities are never far away. A few years ago I found myself back in the Semantic Web space after 10 years (it felt like a family reunion). I'm collaborating with event chairs who were once seeking internships with me. I'm now in my 12th year with Apache (heck, I haven't been with anything for 12 years). If it wasn't for Tim Berners-Lee countering opposition with the belief that "the last thing we need is more of what we already have" I would have never been at W3C; if I hadn't been at W3C I wouldn't have been involved with the Semantic Web or have known Roy Fielding; if I hadn't overcome my not-so-successful first meeting with Roy, he wouldn't have championed me to help launch The Apache Software Foundation. The cycle continues. There's a lot going on out there and it's easy to want to grab it all. What you don't do is just as important as what you do, as it's easy to spread yourself too thin and get overwhelmed. Contribute, but avoid volunteer fatigue. Learn to say "no" in whatever form suits you best. Work hard, but in bursts. Take breaks and clear your desk. Purge that inner critic.

\section*{Expect greatness}
Keep your standards high and know your worth.
