\chapterwithauthor{Kévin Ottens}{University and Community}

\authorbio{Kévin Ottens is a long term hacker of the KDE community. He contributed
to the KDE Platform at large, with a strong emphasis on API design and
frameworks architecture. Graduating in 2007, he holds a PhD in computer science
which led him to work particularly on ontologies engineering and multi-agent
systems. Kévin's job at KDAB includes developing research projects around KDE
technologies. He still lives in Toulouse where he serves as part time teacher in
his former university.}

\section*{Introduction}
Free Culture communities are mostly driven by volunteer efforts. Also
most of the people getting into such communities do so during their time at the
university. It is somewhat the right period of your life to embark in such
adventures: you are young, full of energy, curious, and probably want to change
the world to your image. That is really all that is needed for most volunteer
work.

But, at the same time, being a student does not necessarily leave you plenty of
time to engage with a Free Culture community. Indeed, most of these communities
are rather large, and it can be frightening to contact them.

It obviously raises a scary question: do Free Culture communities, because they
don't try to actively outreach to universities, fail to attract the next generation
of talented contributors?
That is a valid question we tried to explore in the context of a community
producing software, namely KDE. In this article, we focus on the aspects we did
not foresee but had to deal with while looking for an answer to this question.

\section*{Building relationship with a local university}
Really, it all starts by reaching out to the students themselves, and for that,
the best way is still to get to their universities, trying to show them how
welcoming Free Culture communities can be. To that effect, we built a
relationship with the Paul Sabatier University in Toulouse -- more precisely one
of its courses of study named IUP ISI which focused on software engineering.

The IUP ISI was very oriented toward ``hands on'' knowledge, and as such had a
pre-existing program for student projects. A particularly interesting point of
that program is the fact that students work in teams mixing students from
different promotions. Third year and fourth year students get to collaborate on
a common goal, which usually leads to teams of seven to ten students.

The first year of our experiment we hooked up with that program, proposing new
topics for the projects, focusing on software developed within the KDE
community. Henri Massié, director of the course of study, has been very
welcoming to the idea, and let us put the experiment in place. For that first
year, we were allocated two slots for KDE related software projects.

To quickly build trust, we decided that year to provide a few guarantees
regarding the work of the students:
\begin{itemize}
  \item to help the teachers have confidence in the topics covered: the chosen
projects were close to the topics taught at the IUP ISI (that is why we
targeted a UML modeling tool and a project management tool for that year);
  \item to give maximum visibility to the teachers: we provided them a server on
which the student production was regularly built and remotely accessible for
testing purpose;
  \item to ease the engagement of the students with the community: the
maintainers of the projects were appointed to play a ``customer'' role thus
pushing requirements to the students and helping them find their way in the
ramifications of the community;
  \item finally, to get the students going, we introduced a short course on how
to develop with Qt and the frameworks produced by KDE;
\end{itemize}

At the time of this writing, we have been through five years of such projects.
Small adjustments to the organization have been done here and there, but most of
the ideas behind it stayed the same. Most of the changes made were the result of
more and more interest from the community willing to engage with students and
of more and more freedom given to us in the topics we could cover in our
projects.

Moreover, throughout those years the director gave us continuous support and
encouragement, effectively allocating more slots for Free Culture community
projects, proving that our integration strategy was right: building trust very
quickly is key to a relationship between a Free Culture community and a
university.

\section*{Realizing teaching is a two-way process}
During those years of building bridges between the KDE community and the IUP ISI
course of study, we ended up in teaching positions to support the students in
their project related tasks. When you have never taught a classroom
full of students, you might still have this image of yourself sitting in a
classroom a few years ago. Indeed, most teachers were students once... sometimes
not even the type of very disciplined or attentive students. You were likely
having this feeling of drinking from a firehose: a teacher entering a room,
getting in front of the students and delivering knowledge to you.

This stereotype is what most people keep in mind of their years as students and
the first time they get in a teaching situation they want to reproduce that
stereotype: coming with knowledge to deliver.

The good news is that nothing could be further from the truth than this
stereotype. The bad news is that if you try to reproduce it, you are very likely
to scare your students away and face nothing else than lack of motivation on
their side to engage with the community. The image you give of yourself is the
very first thing they will remember of the community: the first time you get in
the classroom, to them \emph{you are} the community!

Not falling into the trap of this stereotype requires you to step back a bit and to realize what teaching is really about. It is not a one way process where one
delivers knowledge to students. We came to the conclusion that it is instead a
two-way process: you get to create a symbiotic relationship with your student.
Both the students and the teacher have to leave the classroom with new
knowledge. You get to deliver your expertise of course -- but to do so efficiently
you have to adapt to the students' frame of reference all the time. It is a very
humbling work.

Realizing that fact generates quite a few changes in the way you undertake
your teaching:
\begin{itemize}
  \item You will have to understand the culture of your students. They probably
have a fairly different background from you and you will have to adapt your
discourse to them; for instance, the students we trained are all part of the so-called Y generation which exhibits fairly different traits regarding leadership, loyalty and trust than previous generations.
  \item You will have to reassess your own expertise, since you will need to
adapt your discourse to their culture. You will approach your own knowledge from
very different angles than what you are used to, which will inevitably lead you
to discoveries in fields you assumed you mastered.
  \item Finally, you will have to build skills in presenting; a teaching
position is really about getting out of your comfort zone to present your own
knowledge while keeping it interesting and entertaining to your audience. It
will make you a better presenter.
\end{itemize}

As such, you will become a better teacher. Also your goals of getting well
trained students, and having students engage with a Free Culture community will
be better fulfilled.

\section*{Conclusion}
At the end of the day why would you go through all the effort
to build trust with a university and step outside of your comfort zone by
improving your teaching? Well, it really boils down to the initial question we
tried to answer:

\emph{Do Free Culture communities fail to attract new contributors out of
universities simply because of their inaction?}

In our experience the answer is \emph{yes}. Through those five years of building
up a relationship with the IUP ISI, we retained around two students per year on
average. Some of them disappeared after a while, but some of them become very
active contributors. The other ones still keep some nostalgia of that period of
their life and keep advocating even though they do not contribute directly. And
right now we have a local KDE team which managed to efficiently organize a two
day conference for our latest release party.

Of those former students, not a single one would have engaged with KDE without
those university projects. We would have completely missed those talents.
Luckily, we have not been inactive.
