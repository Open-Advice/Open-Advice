\chapter{My Project Taught Me How To Grow Up - Runa Bhattacharjee}
\todo{bio}
\section*{Introduction}

Burning the midnight oil has been a favorite form of rebellion by young people
all over the world. Whether to read a book with a torchlight under the covers or
to watch late night TV reruns or (amongst other things) to hang around on an IRC
channel and tinkering around with an itchy problem with a favorite open source
project. 

\section*{How it all began}

That is how it all began for me. Let me first tell a bit about myself. When I
got introduced to the local Linux Users Group in my city, I was in between jobs
and studying for a masters degree. Very soon I was a contributor to a few
localization projects and started translating (mostly) desktop interfaces. We
used a few customized editors with integrated writing methods and fonts. The
rendering engines had not matured well enough to display the script with zero
errors on the interfaces, nonetheless we kept on translating. My focus was on
the workflow that I had created for myself. I used to get the translatable
content from the folks who knew how things work, translate it as best as I
could, add in the comments to help the reviewers understand how I comprehended
the text, filled in the necessary information for copyright and credits and sent
them back to the coordinators.

\section*{How it was done}

It was mostly a simple way of doing things. But most importantly it was
\textit{my} independent way of doing things. I took my own time to schedule when
I would work on the translations. These would then be reviewed and returned to
me for changes. Again, I would schedule them for completion as per how I could
squeeze out some time from all the studying and other work that I was doing.
When I did get down to work, I would sit through 9-10 straight hours mostly into
the wee hours of the morning, feeling a high of accomplishment until the next
assignments came through.

\section*{What mattered}

What I did not know was that I played a significant part in the larger scheme of
things. Namely, release schedules. So, when I completed my 2 cents of the task
and sent them over, I did not factor in a situation where they could be rendered
useless because they were too late for the current release and too early for the
next release (which would invariably contain a lot of changes that would require
a rework). Besides these, I was oblivious to the fact how it all mattered to the
entire release process -- integration, packaging, interface testing, bug filing,
resolution.

\section*{How it made me grow up}

All these changed drastically, when I moved into a more professional role. So
suddenly, I was doing the same thing but in a more structured order. I learned
that the cavalier road-rolling that I had been used to, was not scalable when
one had to juggle through 2-3 release schedules. It had to be meticulously
planned to map with the project roadmaps. While working on translating a desktop
interface, one had to check what the translation schedule was for the main
project. The projected date to start working would be right after when all the
original interface messages had been frozen. Translators could then work
unhindered until the translation deadline, after which they would be marked as
stable in the main repositories and eventually packages would be build. Along
with these schedules, a couple of OS\todo{operating system or open source?}
distributions would align their schedules as well. So the translators had an
additional responsibility of making sure that the pre-release versions of the
operating system that would be carrying the desktop, went through with some bits
of testing to ensure that the translations made sense on the interface and did
not contain errors.

\section*{What I should have known}

The transition was not easy. Suddenly there was a flood of information that I
had to deal with and additional chores that I had to perform. From being a hobby
and more importantly a stress-buster, suddenly it was serious business. Thinking
in retrospect, I can say that it probably helped me understand the entire
process because I had to learn it from ground up. And armed with that knowledge
I can analyze situations with a better understanding of all the effective
facets. At the time when I started working on the Open Source project(s) of my
interest, there were much less professionals who worked full time in this
domain. Most of the volunteer contributors held day jobs elsewhere and saw these
projects as a way to nurture the creative juices that had dried up in their
routine tasks. So a number of newcomers were never mentored about how to plan
out their projects professionally. They grew to be wonderfully skilled in what
they were doing and eventually figured out how they'd like to balance their work
with the rest of the things they were doing.

\section*{Conclusion}

These days I mentor newcomers and one of the first things that I let them know
is how and in which part of the project they matter. Crafting an individual
style of work is essential as it allows a person a comfortable space to work in,
but an understanding of the organized structure that is affected by their work
imbibes the discipline that is required to hold in check chances of arbitrary
caprice.
